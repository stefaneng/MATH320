\documentclass{article}

\usepackage[utf8]{inputenc}
\usepackage[top=1.5in,left=.9in,right=.9in,bottom=1in,headheight=1in]{geometry}
\usepackage{fancyhdr}
\usepackage{lastpage}
\usepackage{amsmath,amssymb,amsthm,mathrsfs}
\usepackage{mathrsfs}
\usepackage{multicol}
\usepackage{enumerate}



% Better looking empty set
\let\emptyset\varnothing

\newtheoremstyle{problem}{\topsep}{\topsep}%
{}%         Body font
{}%         Indent amount (empty = no indent, \parindent = para indent)
{\bfseries}% Thm head font
{\vspace{5pt}}%        Punctuation after thm head
{\newline}%     Space after thm head (\newline = linebreak)
{\thmname{#1}\thmnote{ #3}}%         Thm head spec

\theoremstyle{problem}
\newtheorem{prob}{Problem}

\theoremstyle{plain}
%\newtheorem{thm}{Theorem}
\newtheorem{lem}{Lemma}
\newtheorem{prop}{Proposition}
% No indent for whole page
\setlength\parindent{0pt}

\theoremstyle{remark}
\newtheorem{countex}{Counterexample}

\setlength{\columnsep}{1cm}
%%% Heading -- No need to edit %%%
\pagestyle{fancy}

\rhead{
  Stefan Eng\\
  Dr. Katherine Stevenson\\
  Math 320\\
  11/27/13
}
\lhead{
  Analysis Worksheet
}

\cfoot{Page\ \thepage\ of\ \pageref{LastPage}}

%%%

\renewcommand{\qedsymbol}{$\blacksquare$}
% For manual qed box
\def\bs{\hspace{\stretch1}\ensuremath\blacksquare}

\begin{document}

%%% Make the title %%%
\begin{center}
  \textsc{\Large Foundations of Higher Mathematics}\\[.3cm]
  \textsc{\Large Homework 12 (Analysis 1)}
\end{center}
%%% End title %%%

%%% Start Assignment Here %%%
\begin{enumerate}
  % 1
\item $A = [2,4)$.\\
  % Show 5 is an upper bound
  Observe that $5 \in \mathbb{R}$.
  Take an arbitrary $a \in A$.
  Then, $2 \leq a < 4$, and $4 < 5$, so $a < 5$.
  Thus, $5$ is an upper bound of $A$.
  % Show 5 is not the least upper bound.
  We can see that $4.5 \in \mathbb{R}$, and $a < 4 < 4.5$, so $4.5$ is an upper bound.
  But $4.5 < 5$, so $5$ is not the least upper bound.
  % Prove that 2.1 is not a lower bound  for A
  We can see that $2.05 \in A$, because $2 \leq 2.05 < 4$.
  Also, $2.05 < 2.1$, so $2.1$ is not a lower bound for A.
  % Prove that 4 the least least upper bound for A
  % Show that 4 is an upper bound
  % Show that < 4 is not an upper bound
  Take $a \in A$. Then $2 \leq a < 4$. So $4 \not \in A$, and $a < 4$, so $4$ is an upper bound.
  % 2
\item $S = \{x \in \mathbb{R}|\ \exists n \in \mathbb{N} \text{such that } x = 1 \frac{1}{n}\}$.\\
  Observe that $S \subseteq [0,1)$. % Obvious? or does it need proof
    Take $s \in S$.
    Then, $0 \leq s < 1$.
    We can see that $5 \in \mathbb{R}$ and $5 > 1$, so $s < 1 < 5$.
    Thus, all $s \in S$ are smaller than $5$, so $5$ is an upper bound for $S$.
    Now observe $4 \in \mathbb{R}$, $1 < 4$, so $s < 4$.
    Thus, $4$ is an upper bound as well.
    But $4 < 5$ so $5$ is not the least upper bound.
    % Find the upper bound
    We know that $S > 1$.
    Call $S - 1 = \epsilon > 0$
    By the archimedian principle, there exists an $n \in \mathbb{N}$ such that $\frac{1}{n} < \epsilon$.
    % ...
    % ...
    

  % 3
\item 
  % a
  Assume $a,b \in \mathbb{R}$ with $a < b$ and $a \geq 0$. Thus, $a - b > 0$.
  % b
  By the archimedian property, there is an $n \in \mathbb{N}$ such that $\frac{1}{n} < b - a$.
  % c
  It follows that:
  \begin{align*}
    \frac{1}{n} &< b - a\\
    \frac{1}{n} + a &< b\\
    a &< b - \frac{1}{n}
  \end{align*}
  % d
  Let $m$ be the smallest number such that $m > na$, 
  % BY WHAT? I don't think this is right
  by the archimedian property, since $na \in \mathbb{R}_{>0}$
  % e
  Since $m$ is the smallest integer greater than $na$, then $m - na \leq 0$
  ...
  % 4
\item % 
  % a  
  Let $T = \{t \in \mathbb{R}|\ t^2 < 2\}$.
  Take $0 \in \mathbb{R}$. 
  We can see that $0^2 < 2$, so $0 \in \mathbb{R}$. 
  Thus, $T$ is non-empty.
  Observe that $2 \in \mathbb{R}$.
  $2^2 \not < 2$, so $2 \not \in T$.
  Also, $0 < 2$, so $2$ is an upper bound of $T$. 
  Call $U$ the set of all upper bounds for $T$.
  Since $2 \in U$, then $U$ is non-empty.
  By the well-ordering principle, there must be a least element of $U$.
  Call this least element $x$.
  % b
  Now we want to show that $x^2 = 2$, by showing the $x^2 \not < 2$ and $x^2 \not > 2$.
  % c
  % Aiming for contradiction
  Assume $x^2 < 2$.
  % d
  Take $n \in \mathbb{N}$ such that, $(x + \frac{1}{n})^2 = x^2 + \frac{2x}{n} + \frac{1}{n^2}$.
  This yields the inequality $x^2 + \frac{2x}{n} + \frac{1}{n^2} \leq x^2 + \frac{2x+1}{n}$, because
  \begin{align*}
    x^2 + \frac{2x}{n} + \frac{1}{n^2} &\leq x^2 + \frac{2x+1}{n}\\
    \frac{2x}{n} + \frac{1}{n^2} \leq \frac{2x+1}{n}\\
    \frac{1}{n^2} \leq 1
  \end{align*}  
  % e
  Since $x^2 < 2$, then $0 < 2 - x^2$.
 % Also, $2x + 1 > 0$, because
 % \begin{align*}
 %   2x + 1 &> 0\\
 %   2x > -1\\
 %   x > -\frac{1}{2}\\
 %   x^2 > \frac{1}{4}
 % \end{align*}
 % Which is clearly true since $x^2 < 2$.
  % f
  By the archimedian principle, there exists an $n_0 \in \mathbb{N}$ such that $\frac{1}{n_0} < \frac{2 - x^2}{2x+1}$.
  % g
  Multiplying both sides by $2x+1$, we get $\frac{2x + 1}{n_0} < 2 - x^2$.
  % h
  Thus, $x^2 + \frac{2x+1}{n_0} < 2$.
  % i
  As was shown earlier, $(x + \frac{1}{n})^2 = x^2 + \frac{2x}{n} + \frac{1}{n^2} \leq x^2 + \frac{2x + 1}{n}$.
  So, (x + \frac{1}{n_0})^2 < 2.
  % j
  Thus, $x + \frac{1}{n_0} \in T$, by definition of our set $T$.
  This contradicts the fact that $x$ is the least upper bound, because $x + \frac{1}{n_0} > x$, and anything largest than the least upper bound is not in $T$. 
  So, $x^2 \not < 2$.
  % k
  Now assume that $x^2 > 2$
  It follows that $x^2 - 2 > 0$.
  % Need an argument that $\frac{x^2 - 2}{2x} > 0$,
  Since $\frac{x^2 - 2}{2x} > 0$, by the archimedian principle, there exists $m_0 \in \mathbb{N}$, such that $\frac{1}{m_0} < \frac{x^2 - 2}{2x}$
  % l
  Multiplying both sides by $2x$ leaves us with, $\frac{2x}{m_0} < x^2 - 2$.
  Thus, $2 < x^2 - \frac{2x}{m_0}$.
  % m
  Now, observe that $(x - \frac{1}{m_0})^2 = x^2 - \frac{2x}{m_0} + \frac{1}{m_0} > x^2 - \frac{2x}{m_0}.
  It follows that $2 < (x - \frac{1}{m_0})^2$.
  % n
  By our definition for $T$, $x - \frac{1}{m_0} \not \in T$. But $x - \frac{1}{m_0} < x$, which contraditcts the fact that $x$ is the least upper bound.
  Thus, $x^2 \not > 2$.
  
  % 
  Since $x^2 \not < 2$ and $x^2 \not > 2$, $x^2 = 2$. 
  
\end{enumerate}

\end{document}
