\documentclass{article}

\usepackage[utf8]{inputenc}
\usepackage[top=1.5in,left=1in,right=1in,bottom=1.5in,headheight=1in]{geometry}
\usepackage{fancyhdr}
\usepackage{amsmath,amsthm,amsfonts,amssymb}
\usepackage{setspace}
\usepackage{multicol}
\usepackage{float}

% \theoremstyle{theorem}
\newtheorem*{thm}{Theorem}
\newtheorem*{lemma}{Lemma}
\renewcommand{\qedsymbol}{$\blacksquare$}

%%% Heading -- No need to edit %%%
\pagestyle{fancy}
\rhead{
Stefan Eng \\ 
Dr. Katherine Stevenson \\ 
Math320 \\ 
10/28/13
}
%%%

\begin{document}
\pagenumbering{gobble}

%%% Make the title %%%
\begin{center}
\textsc{\Large Foundations of Higher Mathematics}\\[.3cm]
\textsc{\Large Journal 7}\\[1cm]
\end{center}

%%% End title %%%

%%% Start Journal Here %%%
\section*{Cardinality and Equivalence Relations}
Since we just finished equivalence relations, and we are going to study functions, and I have done a few of my journals on cardinality, it seemed like a nice topic to write about. The goal is to show that two sets having the same cardinality is an equivalence relation. Using a less formal definition, cardinality is the one-to-one pairing of elements from one set to the other. A few definitions we need are \textit{surjective}, \textit{injective} and the inverse. Given a function $f : X \to Y$, the inverse (if it exists) is defined as a function $f^{-1} : Y \to X$. A surjective function is on that for every $y$ is $Y$, there is an $x$ such that $f(x) = y$. An injective function is one such that every element in $X$ has a unique corresponding element in $Y$. That is, $\forall a,b \in X, f(a) = f(b)$ implies $a = b$. When a function $X \to Y$ is surjective and injective, we say the function is \textit{bijective}. The definition of cardinality can be defined with this definition. A set $A$ is said to have the same cardinality as a set $B$ is there exists a bijection from $A$ to $B$. Symbolically, $|A| = |B|$. If this is true, $A$ and $B$ are said to be equinumerous. Finally after all of these definitions we get to the theorem (Some of the details are glossed over).
\begin{thm}
  $|A| = |B|$ is an equivalence relation.
\end{thm}
\begin{proof}
  \textbf{(Reflexive)} Take the identity function $id : A \to A$. Since the inverse of the identity function is the identity function, $A \to A$ have a bijection. Stated in the above paragraph, a bijection implies $|A| = |A|$.\\
  \textbf{(Symmetric)} Suppose we have $|A| = |B|$. From our definition we have that $f : A \to B$ is a bijection. A bijection implies that there exists and inverse, $f^{-1} B \to A$. Since $a \in A; f(f^{1}(a)) = a$, (The fact that $f(f^{-1}(x)) = x$ comes from that $f$ is a bijection and has a one-to-one correspondence). and $b \in B; f^{-1}(f(b)) = b$. It follows that $f^{-1}$ is a bijection, so $|B| = |A|$.\\
  \textbf{(Transitive)} If we have a sets $A$, $B$ and $C$ such that $|A| = |B|$ and $|B| = |C|$, we have functions $f : A \to B$ and $g: B \to C$ where $f$ and $g$ are bijections. This will be a similar approach to symmetry. The goal is to show that $(g \circ f)^{-1}$ is the inverse of $g \circ f$, and having an inverse implies that a function is a bijection. So, $f^{-1} : B \to A$, because $f$ is a bijection, $g^{-1} : C \to B$ because $g$ is also a bijection. It follows that, 
  \begin{align*}
    ((f^{-1} \circ g^{-1}) (g \circ f)) (a) &= f^{-1}(g^{-1}(g(f(a)))) && \text{where }a \in A\\
    &= f^{-1}(f(a))\\
    &= a
  \end{align*}
  It similarly follows for $((g \circ f) \circ (f^{-1} \circ g^{-1})) (c) = c$, where $c \in C$. So, $f^{-1} \circ g^{-1}$ is the inverse of $g \circ f$, and $f^{-1} \circ g^{-1} = (g \circ f)^{-1}$
\end{proof}

\section*{Conclusion}
Once I started along this proof I realized that there are lots of theorems and lemmas that I had to omit and skim over. The gist of the proof is interesting but may have been better to do after we went over functions. Anyway, it was fun to connect the journals I had done previously with the current and future topics in class.

\end{document}

%%% End assignment %%%

