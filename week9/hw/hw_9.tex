\documentclass{article}

\usepackage[utf8]{inputenc}
\usepackage[top=1.5in,left=.9in,right=.9in,bottom=1in,headheight=1in]{geometry}
\usepackage{fancyhdr}
\usepackage{lastpage}
\usepackage{amsmath,amssymb,amsthm}
\usepackage{mathrsfs}
\usepackage{multicol}
\usepackage{enumerate}



% Better looking empty set
\let\emptyset\varnothing

\newtheoremstyle{problem}{\topsep}{\topsep}%
{}%         Body font
{}%         Indent amount (empty = no indent, \parindent = para indent)
{\bfseries}% Thm head font
{\vspace{5pt}}%        Punctuation after thm head
{\newline}%     Space after thm head (\newline = linebreak)
{\thmname{#1}\thmnote{ #3}}%         Thm head spec

\theoremstyle{problem}
\newtheorem{prob}{Problem}

\theoremstyle{plain}
%\newtheorem{thm}{Theorem}
\newtheorem{lem}{Lemma}
\newtheorem{prop}{Proposition}
% No indent for whole page
\setlength\parindent{0pt}

\theoremstyle{remark}
\newtheorem{countex}{Counterexample}

\setlength{\columnsep}{1cm}
%%% Heading -- No need to edit %%%
\pagestyle{fancy}

\rhead{
  Stefan Eng\\
  Dr. Katherine Stevenson\\
  Math 320\\
  10/30/13
}
\lhead{
  Ch 4: 66, 67, 68, 70, 73, 75, 81
}

\cfoot{Page\ \thepage\ of\ \pageref{LastPage}}

%%%

\renewcommand{\qedsymbol}{$\blacksquare$}
% For manual qed box
\def\bs{\hspace{\stretch1}\ensuremath\blacksquare}

\begin{document}

%%% Make the title %%%
\begin{center}
\textsc{\Large Foundations of Higher Mathematics}\\[.3cm]
\textsc{\Large Homework 9}
\end{center}
%%% End title %%%

%%% Start Assignment Here %%%
\begin{prob}[66]


\end{prob}
%

\begin{prob}[67]

\end{prob}
%

\begin{prob}[68]
($\impliedby$) If $n$ dividing $a$ and $n$ dividing $b$ have the same remainder, then $a = cn + r$ and $b = dn + r$. It follows that:
\begin{align*}
  a - b &= (cn + r) - (dn + r)\\
  &= cn - dn\\
  &= n(c -d)
\end{align*}
It follows that $n\big | a - b$ and thus $a \equiv b \pmod{n}$.\\

($\implies$) Assume $a \equiv b \pmod{n}$. Then $n \big | a - b$. It follows that $nk = a - b$. Apply the division algorithm on $n$ and $b$. It follows that $b = nq + r$.
\begin{align*}
  nk &= a - b\\
  a &= nk + b\\
  a &= nk + (nq + r)\\
  a &= n(k+q) + r
\end{align*}
We see that both $a$ and $b$ have the same remainder $r$.
\end{prob}
%

\begin{prob}[70]\ \\[-1cm]
  \begin{proof}
    Let $m = 1$. If $a \equiv b \pmod{n}$, then $a^1 \equiv b^1 \pmod{n}$. So $a^m \equiv b^m \pmod{n}$. Thus, $m = 1$ is a solution. Now assume that if $a \equiv b \pmod{n}$ then $a^m \equiv b \pmod{n}$. Assume $a \equiv b \pmod{n}$. It follows that $a^m \equiv b^m \pmod{n}$. From Theorem 4.10: $a \cdot a^m \equiv b \cdot b^m \pmod{n}$. Thus, $a^{m+1} \equiv b^{m+1} \pmod{n}$.
  \end{proof}
\end{prob}
% 

\begin{prob}[73]

\end{prob}
%

\begin{prob}[75]

\end{prob}
%

\begin{prob}[81]

\end{prob}
%


\end{document}