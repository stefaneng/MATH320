\documentclass{article}

\usepackage[utf8]{inputenc}
\usepackage[top=1.5in,left=.9in,right=.9in,bottom=1in,headheight=1in]{geometry}
\usepackage{fancyhdr}
\usepackage{lastpage}
\usepackage{amsmath,amssymb,amsthm}
\usepackage{mathrsfs}
\usepackage{multicol}
\usepackage{enumerate}
%\usepackage{concmath}
%\usepackage[T1]{fontenc}


% Better looking empty set
\let\emptyset\varnothing

\newtheoremstyle{problem}{\topsep}{\topsep}%
{}%         Body font
{}%         Indent amount (empty = no indent, \parindent = para indent)
{\bfseries}% Thm head font
{\vspace{5pt}}%        Punctuation after thm head
{\newline}%     Space after thm head (\newline = linebreak)
{\thmname{#1}\thmnote{ #3}}%         Thm head spec

\theoremstyle{problem}
\newtheorem{prob}{Problem}

\theoremstyle{plain}
%\newtheorem{thm}{Theorem}
\newtheorem{lem}{Lemma}
\newtheorem{prop}{Proposition}
% No indent for whole page
\setlength\parindent{0pt}

\theoremstyle{remark}
\newtheorem{countex}{Counterexample}

\setlength{\columnsep}{1cm}
%%% Heading -- No need to edit %%%
\pagestyle{fancy}

\rhead{
  Stefan Eng\\
  Dr. Katherine Stevenson\\
  Math 320\\
  10/9/13
}
\lhead{
  Ch 3: \# 84, 94, 95, 101, 103, 111, 112 118
}

\cfoot{Page\ \thepage\ of\ \pageref{LastPage}}
%%%

\renewcommand{\qedsymbol}{$\blacksquare$}
% For manual qed box
\def\bs{\hspace{\stretch1}\ensuremath\blacksquare}

\begin{document}

%%% Make the title %%%
\begin{center}
\textsc{\Large Foundations of Higher Mathematics}\\[.3cm]
\textsc{\Large Homework 6}
\end{center}
%%% End title %%%

%%% Start Assignment Here %%%
\begin{prob}[84]\ \\[-1cm]
\begin{enumerate}[a)]
\item a = 901, b = 952
  \begin{align*}
    952 &= 901(1) + 51\\
    901 &= 51(17) + 35\\
    51 &= 34(1) + 17\\
    34 &= 17(2) + 0   
  \end{align*}
  So, gcd$(901,952) = 17$.
  \begin{align*}
    17 &= 51 - 34\\
       &= 51 - [901 - 51(17)]\\
       &= 51(18) - 901\\
       &= (952-901)(18) - 901\\
       &= 952(18) - 901(19)
  \end{align*}
\item a = 4199, b = 1748
  \begin{align*}
    4199 &= 1748(2) + 703\\
    1748 &= 703(2) + 342\\
    703 &= 342(2) + 19\\
    342 = 19(18)
  \end{align*}
So, gcd(4199,1748) is 19.
\begin{align*}
  19 &= 703 - 342(2)\\
  &= 703 - [1748 - 703(2)](2)\\
  &= 703 - [2(1748) - 4(703)]\\
  &= (-2)1748 + 5(703)
\end{align*}

\item a = 377, b = 233
\begin{align*}
  377 &= 233(1) + 144\\
  233 &= 144(1) + 89\\
  144 &= 89(1) + 55\\
  89 &= 55(1) + 34\\
  55 &= 34(1) + 21\\
  34 &= 21(1) + 13\\
  21 &= 13(1) + 8\\
  13 &= 8(1) + 5\\
  8 &= 5(1) + 3\\
  5 &= 3(1) + 2\\
  3 &= 2(1) + 1\\
  2 &= 1(1) + 1\\
  1 &= 1(1) + 0
\end{align*}
So, gcd(377,233) = 1.
$$
1 = 377(322)+233(-521)
$$
\end{enumerate}
\end{prob}

%
\begin{prob}[94]
Suppose $a$ and $b$ are non-zero integers.
\begin{enumerate}
\item To show that there exists a number $m$ such that $a|m$ and $b|m$, we can define a set $\{n \in \mathbb{N} |\ a|n\text{ and }b|n \}$. We can see that $ab$ is in our set since $a|ab$ and $b|ab$. Since our set is non-empty it has a least element m, by the least natural number principle. Thus, $a|m$ and $b|m$.
\item It follows from part 1 that $ak_0 = m$ and $bj_0 = m$. Assume that $c \in \mathbb{Z}$, such that $a|c$ and $b|c$. Thus, $ak_1 = c$ and $bj_1 = c$. We want to show that $m$ divides $c$. From the division algorithm we can see that $c = mq + r$, $0 \leq r < m$. It follows that:
\end{enumerate}
\begin{align*}
  c &= mq + r\\
  r &= c - mq\\
    &= ak_1 - (ak_0)q\\
    &= bj_1 - (bj_0)q
\end{align*}
So, $a|r$ and $b|r$. Since $r < m$, $r$ must be zero because m is the least element that both $a$ and $b$ divide. Therefore, $m|c$.
\end{prob}

%
\begin{prob}[95]
Let's assume that $m$ and $m'$ satisfy conditions 1 and 2. Then $a|m'$ and $b|m'$ and $a|m$ and $b|m$. By condition 2 we have that $m|m'$ by replacing $c$ by $m'$ since it is a divisor or $a$ and $b$. Similarly, $m'|m$. Therefore, $m = m'$ proving that the least common divisor is unique.
\end{prob}

%
\begin{prob}[101]
Let $S = \{n \in \mathbb{N} |\ \exists p : p|a_i \text{ given that }p|a_1a_2\ldots a_n \}$. If $n = 1$, we have $p|a_1$, so clearly $p|a_i$, where $i = 1$. %Thus, $1 \in S$. 
Assume that $n \in S$. We want to show that $n + 1 \in S$. Thus, $p | a_1 a_2 \ldots a_n a_{n+1}$. From the induction hypothesis there exists $i$ such that $p|a_i$. So given that $p|a_i$ and $p|a_1 a_2 \ldots a_i \ldots a_{n+1}$ we see that $a_i$ from the induction hypothesis is still there so we are done. There is $a_i$ such that $p|a_i$.
\end{prob}

%
\begin{prob}[103]
Assume $a \not = 0$, $b \not = 0$ and $d \in \mathbb{N}$ such that, $d|a$ and $d|b$.
\begin{enumerate}
\item[($\Rightarrow$)] Suppose that gcd$(a,b) = d$. It follows that there exist $m,n$ such that $ma + nb = d$. Since $d|a$ and $d|b$, we can see that $dk=a$ and $dj=b$. Thus, 
\begin{align*}
m(dk) + n(dj) &= d\\
d(mk + nj) &= d\\
mk + nj &= 1\\
\end{align*}
Since $k = \frac{a}{d}$ and $j = \frac{b}{d}$ (Since $d|a$ and $d|b$ it is fine to have a fraction, it will be in $\mathbb{Z}$),  $gcd(\frac{a}{d},\frac{b}{d}) = 1$.
\item[($\Leftarrow$)] Suppose gcd$(\frac{a}{d}, \frac{b}{d}) = 1$. Then there exist $m,n$ such that $m(\frac{a}{d}) + n(\frac{b}{d}) = 1$. It follows that 
\begin{align*}
  d[m(\frac{a}{d}) + n(\frac{b}{d}) &= 1]\\
  ma + nb &= d
\end{align*}
Therefore, $gcd(a,b) = d$.
\end{enumerate}
\end{prob}

%
\begin{prob}[111]
Let $p,q,r \in \mathbb{Z}$ such that 5 divides $p^2 + q^2 + r^2$. Assume not. Suppose p, q, and r do not divide 5. We can express this as $p = 5k_0 + x_0$, $q = 5k_1 + x_1$, $r = 5k_2 + x_2$ where $1 \leq x < 5$. It follows that:
\begin{align*}
  (5k_0 + x_0)^2 + (5k_1 + x_1)^2 + (5k_2 + x_2)^2 =\\
(25k_0^2 + 10k_0x_0 + x_0^2) + (25k_1^2 + 10k_1x_1 + x_1^2) + (25k_2^2 + 10k_2x_2 + x_2^2)= \\
25(k_0^2 + k_1^2 + k_2^2) + 10(k_0x_0+k_1x_1+k_2x_2) + (x_0^2 + x_1^2 + x_2^2)
\end{align*}
Since $1 \leq x_0, x_1, x_2 < 5$, we can see that 5 does not divide $p^2 + q^2 + r^2$. This is a contradiction so 5 must divide at least one of $p,q,r$. 
\end{prob}

%
\begin{prob}[112]
Assume not. Suppose that $p_1,p_2,\ldots,p_i$ are the only primes of the form, $4n+3$. Let $N = p_1 p_2 \ldots p_i -1$. We can see that $4N + 3$ is $4(p_1 p_2 \ldots p_i - 1) + 3 = 4p_1 p_2 \ldots p_i- 1$. We can see that every prime in our list of primes in the form $4n+3$ leaves a remainder of 4. Thus, $4N + 3$ is prime which contradicts our statement that $p_1,p_2,\ldots ,p_i$ are the only primes of the form $4n+3$. Therefore, there are infinitely many primes of the form $4n+3$.

\end{prob}

%
\begin{prob}[118]
\begin{enumerate}[a)]
\item Since the gcd(17,13) = 1, there are no solutions, by theorem 3.16.
\item The gcd(21,14) = 7 and $7|147$ so there are infinitely many solutions. One solutions is: $21(11) + 14(-6) = 147$. It follows that all the solutions are in the form:
\begin{align*}
x &= 11 + \frac{14}{7}k\\
y &= -6 - \frac{21}{7}k
\end{align*}
\item The gcd(60,18) = 6 which does not divide 97 so there are no integral solutions.
\item The gcd(738,621) = 9. Since $9|45$, there are infinitely many solutions. One of these solutions is $738(11) + 621(-12) = 45$ So from theorem 3.16 the solution sets are:
\begin{align*}
  x &= 11 + \frac{621}{9}\\
  y &= -12 + \frac{638}{9}
\end{align*}
\end{enumerate}
\end{prob}




\end{document}