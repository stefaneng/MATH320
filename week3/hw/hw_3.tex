\documentclass{article}

\usepackage[utf8]{inputenc}
\usepackage[top=1.5in,left=1in,right=1in,bottom=1.5in,headheight=1in]{geometry}
\usepackage{fancyhdr}
\usepackage{lastpage}
\usepackage{amsmath,amssymb,amsthm}

% Better looking empty set
\let\emptyset\varnothing

\newtheoremstyle{problem}{\topsep}{\topsep}%
{}%         Body font
{}%         Indent amount (empty = no indent, \parindent = para indent)
{\bfseries}% Thm head font
{}%        Punctuation after thm head
{\newline}%     Space after thm head (\newline = linebreak)
{\thmname{#1}\thmnote{ #3}}%         Thm head spec

\theoremstyle{problem}
\newtheorem{prob}{Problem}

% No indent for whole page
\setlength\parindent{0pt}

%%% Heading -- No need to edit %%%
\pagestyle{fancy}

\rhead{
  Stefan Eng\\
  Dr. Katherine Stevenson\\
  Math 320\\
  9/18/13
}
\lhead{
  (1.4-1.5) 63, 66, 70, 72, 75\\
  (2.2) 24, 29a, c, d, 30, 31
}

\cfoot{Page\ \thepage\ of\ \pageref{LastPage}}
%%%

\renewcommand{\qedsymbol}{$\blacksquare$}
% For manual qed box
\def\bs{\hspace{\stretch1}\ensuremath\blacksquare}

\begin{document}

%%% Make the title %%%
\begin{center}
\textsc{\Large Foundations of Higher Mathematics}\\[.3cm]
\textsc{\Large Homework 3}
\end{center}
%%% End title %%%

%%% Start Assignment Here %%%

\section*{Chapter 1}
%%%%%%%%%%%%%%%%%%%%%%%%%%%%%%%%%%%%%%%%%%%%%%%%%%%
\begin{prob}[63]
  Let A be an even integer and B be an odd integer. Prove that A + B is odd and AB is even.
\begin{proof}
    If A is even and B is odd, then there exists integers k, n, such that $A = 2q$ and $B = 2k + 1$. It follows that, 
    \begin{align*}
      A + B &= 2q + (2k + 1)\\
            &= (2q + 2k) + 1
    \end{align*}
    Thus, since $2q + 2k \in \mathbb{Z}$, A + B is odd. Similarly for AB,
    \begin{align*}
      AB &= 2q(2k+1)\\
      &= 4qk + 2q\\
      &= 2(2qk + q)
    \end{align*}
    it follows that since $2qk+q$ is an integer, AB is even.
  \end{proof}
\end{prob}
%%%%%%%%%%%%%%%%%%%%%%%%%%%%%%%%%%%%%%%%%%%%%%%%%%%

\begin{prob}[66]
  Let $x$ be a real number. Prove $x = -1$ if and only if $x^3 + x^2 + x + 1 = 0$
  \renewcommand\qedsymbol{}
  \begin{proof} ($\rightarrow$)\\
    Assume $x = -1$. Then 
    \begin{align*}
      x^3 + x^2 + x + 1          &= 0\\
      (-1)^3 + (-1)^2 + (-1) + 1 &= 0\\
      -1 + 1 -1 + 1              &= 0\\
      0                          &= 0
    \end{align*}
      Thus, $x = -1$ implies $x^3 + x^2 + x + 1 = 0$.
  \end{proof}

  \begin{proof} ($\leftarrow$)
    Assume $x^3 + x^2 + x + 1 = 0$. Then
    \begin{align*}
      x^3 + x^2 + x + 1     &= 0\\
      (x^3 + x^2) + (x + 1) &= 0\\
      x^2(x+1) + (x+1)      &= 0\\
      (x+1)(x^2 + 1)        &= 0
    \end{align*}
    Since $x \in \mathbb{R}$ the only solution is $x = -1$. Therefore, $x^3 + x^2 + x + 1 = 0$ implies $x = -1$.
  \end{proof}
Proving both directions, we can conclude that $x = -1$ if and only if $x^3 + x^2 + x + 1 = 0 \bs$
\end{prob}
\renewcommand{\qedsymbol}{$\blacksquare$}
%%%%%%%%%%%%%%%%%%%%%%%%%%%%%%%%%%%%%%%%%%%%%%%%%%%

 \begin{prob}[70]
   Let $n$ be an integer such that $n^2$ is even. Prove that $n^2$ is divisible by 4.
   \begin{proof}
     Assume $n^2$ is even. Then by theorem 1.12 in the book, $n$ is even. Since $n$ is even, there exists a $k$ such that, $n = 2k$. It follows that,
     \begin{align*}
       n^2 &= 2k \times 2k\\
           &= 4k^2\\
           &= 4(k^2)
     \end{align*}
     Since $k^2$ is an integer, 4 divides $n^2$.
  \end{proof}
\end{prob}
 %%%%%%%%%%%%%%%%%%%%%%%%%%%%%%%%%%%%%%%%%%%%%%%%%%%

\begin{prob}[72]
The student was 
\end{prob}

\begin{prob}[75]
  Prove, by contradiction, that the sum of two even integers is even.
\begin{proof}\textit{(Contradiction)}: Assume the sum of two even integers is odd. Then there exist integers $k,q$ such that, 2k + 2q = 2(k + q). This is a contradiction, since we said the sum was odd, so therefore the sum of two even integers is even.
\end{proof}

\end{prob}

\section*{Chapter 2}
\begin{prob}[24]
Prove that $A \cap B = A$ if and only if $A \subseteq B$.
\renewcommand\qedsymbol{}
\begin{proof}($\rightarrow$)
Assume $A \cap B = A$. Then $A \subseteq A \cap B$. If we take an arbitrary element $x$ of $A$, then $x$ is in $A$ and $B$. It follows that every element in $A$ is also in $B$. Thus, $A \subseteq B$. Therefore, $A \cap B = A$ implies $A \subseteq B$.
\end{proof}
\begin{proof}($\leftarrow$)
Assume $A \subseteq B$. Suppose $x$ is an arbitrary element of $A$. Then $x$ is in $B$. So if $x$ is in $A$ and $x$ is in $B$, then for all $x$ that are in $A$ and $B$, $x$ is in $A$. Thus, $A \cap B \subseteq A$. Similarly, $A \subseteq A \cap B$ which implies that $A \cap B = A$. Therefore, $A \subseteq B$ implies $A \cap B = A$.
\end{proof}
\renewcommand{\qedsymbol}{$\blacksquare$}
Therefore, $A \cap B = A$ if and only if $A \subseteq B\bs$
\end{prob}

\begin{prob}[29a]
$\emptyset \cap A = \emptyset$ and $\emptyset \cup A = A$
\begin{proof}($\emptyset \cap A = \emptyset$)
Assume an arbitrary $x$ in $\emptyset$ and A. There are no elements that are in $\emptyset$ and $A$ so it is (vacuously) true that x $\in \emptyset$. The statement
$\emptyset \subseteq \emptyset \cap A$ follows directly from the fact the empty set is a subset of every set. Therefore $\emptyset \cap A = \emptyset$.
\end{proof}
\begin{proof}($\emptyset \cup A = A$)
Assume $x$ is in $\emptyset$ or $A$. If $x$ in $A$, $A \subseteq A$ from \textit{proposition 2.2} in the book. If $x$ in $A$, $\emptyset \subseteq A$ since $\emptyset$ is a subset of every set. Also, if $x$ in $A$, then $x$ in $A$ or $\emptyset$. Thus, $\emptyset \cup A \subseteq A$ and $A \subseteq \emptyset \cup A$. Therefore, $\emptyset \cup A = A$.
\end{proof}
\end{prob}

\begin{prob}[29c]
$A \subseteq A \cup B$
\begin{proof}
Assume $x$ is an element of $A$. Then $x$ is in $A$ or $x$ is in $B$. Therefore, $A \subseteq A \cup B$.
\end{proof}
\end{prob}

\begin{prob}[29d]
$A \cup B = B \cup A$ and $A \cap B = B \cap A$
\begin{proof}($A \cup B = B \cup A$)\\
Suppose x is and element of $A \cup B$. Then x is in $A$ or x is in $B$. Thus $x$ is in $B$ or $x$ is in $A$. Hence, $A \cup B \subseteq B \cup A$ and similarly $B \cup A \subseteq A \cup B$. Therefore, $A \cup B = B \cup A$.
\end{proof}
\begin{proof}($A \cap B = B \cap A$)\\
The proof is the same as above, replacing the words \textit{or} for \textit{and}.
\end{proof}
\end{prob}

\begin{prob}[30]
\textbf{Counterexample:} $P = \emptyset, Q = \emptyset, R = \{1\}$
\begin{align*}
  (P \cap Q) \cup R                     &= P \cap (Q \cup R)\\
  (\emptyset \cap \emptyset) \cup \{1\} &= \emptyset \cap (\emptyset \cup \{1\})\\
  \emptyset \cup \{1\}                  &= \emptyset \cap \{1\}\\
  \{1\}                                 &\not = \emptyset
\end{align*}
\end{prob}

\begin{prob}[31]
The student's proof was \textbf{wrong.} The mistake he or she made was in the step, $x \in A$ and $x \not \in B$ or $x \not \in C$. The statement should look like, $x \in A$ and $x \not \in B$ \textbf{and} $x \not \in C$.

\end{prob}
%%% End assignment %%%
\end{document}



