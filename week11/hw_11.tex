\documentclass{article}

\usepackage[utf8]{inputenc}
\usepackage[top=1.5in,left=.9in,right=.9in,bottom=1in,headheight=1in]{geometry}
\usepackage{fancyhdr}
\usepackage{lastpage}
\usepackage{amsmath,amssymb,amsthm}
\usepackage{mathrsfs}
\usepackage{multicol}
\usepackage{enumerate}



% Better looking empty set
\let\emptyset\varnothing

\newtheoremstyle{problem}{\topsep}{\topsep}%
{}%         Body font
{}%         Indent amount (empty = no indent, \parindent = para indent)
{\bfseries}% Thm head font
{\vspace{5pt}}%        Punctuation after thm head
{\newline}%     Space after thm head (\newline = linebreak)
{\thmname{#1}\thmnote{ #3}}%         Thm head spec

\theoremstyle{problem}
\newtheorem{prob}{Problem}

\theoremstyle{plain}
%\newtheorem{thm}{Theorem}
\newtheorem{lem}{Lemma}
\newtheorem{prop}{Proposition}
% No indent for whole page
\setlength\parindent{0pt}

\theoremstyle{remark}
\newtheorem{countex}{Counterexample}

\setlength{\columnsep}{1cm}
%%% Heading -- No need to edit %%%
\pagestyle{fancy}

\rhead{
  Stefan Eng\\
  Dr. Katherine Stevenson\\
  Math 320\\
  11/20/13
}
\lhead{
  Chapter 5 67, 69, 70, 72, 75
}

\cfoot{Page\ \thepage\ of\ \pageref{LastPage}}

%%%

\renewcommand{\qedsymbol}{$\blacksquare$}
% For manual qed box
\def\bs{\hspace{\stretch1}\ensuremath\blacksquare}

\begin{document}

%%% Make the title %%%
\begin{center}
\textsc{\Large Foundations of Higher Mathematics}\\[.3cm]
\textsc{\Large Homework 11}
\end{center}
%%% End title %%%

%%% Start Assignment Here %%%
\begin{prob}[67] $f: \mathbb{R} \to \mathbb{R}$ such that $f(x) = 3x^2 + 2$.
\begin{enumerate}[a)]
\item $f([2,3])$\\
$f(2) = 3(2)^2 + 2 = 14$ and $f(3) = 3(3)^2 + 2 = 29$.
So, $f([2,3]) = [14,29]$.
\item $f^{-1}([55,307]) = \{ x \in \mathbb{R}|\ f(x) \in [55,307]\}$\\
\begin{multicols}{2}
\begin{align*}
  55 &= 3x^2 + 2\\
  53 &= 3x^2\\
  \frac{53}{3} &= x^2\\
  \pm \sqrt{\frac{53}{3}} &= x
\end{align*}
\vfill
\columnbreak
\begin{align*}
  307 &= 3x^2 + 2\\
  305 &= 3x^2\\
  \pm \sqrt{\frac{305}{3}} &= x
\end{align*}
\end{multicols}
So, $f^{-1}([55,307]) = \left [-\sqrt{\frac{53}{2}}, \sqrt{\frac{305}{5}}\  \right ] $
% c
\item $f^{-1}([1,2]) = \{x \in \mathbb{R}|\ f(x) \in [1,2]\}$\\
\begin{align*}
  1 \leq 3x^2 + 2 \leq 2\\
  -1 \leq 3x^2 \leq 0
\end{align*}
Since $3x^2 = -1$ has no solutions in $\mathbb{R}$, $f^{-1}([1,2]) = \{0\}$.
% d
\item \begin{align*}
    f^{-1}(f(3)) &= f^{-1}(3\cdot (3)^2 + 2)\\
    &= f^{-1}(29)\\
    29 &= 3x^2 + 2\\
    27 &= 3x^2\\
    x &= \pm 3
  \end{align*}  

% e
\item $f(\{-1, -2, -3\}) = \{3(-1)^2 + 2, 3(-2)^2 + 2, 3(-2)^2 + 2\} = \{5, 14, 29\}$

% f
\item $f(\{1, 2, 3\}) = \{5, 14, 29\}$, same work as last problem.
\end{enumerate}
\end{prob}

\begin{prob}[69]
\begin{enumerate}[a)]\ \\[-1cm]
% a
\item $f(A) = \{mn|\ (m,n) \in \mathbb{N} \times \mathbb{N}: m = 1 \text{ and } n \text{ is even}\}$\\
$= 1\cdot n$ where $n$ is even, so $f(A)$ is all even numbers.
% b
\item $f(B) = \{mn|\ (m,n) \in \mathbb{N} \times \mathbb{N}: m \text{ and } n \text{ are even}\}$\\
If $m$ and $n$ are even then, $m = 2j$ and $n = 4k$, so $mn = 4jk$. Thus, $f(B) = 4\mathbb{N}$.
% c
\item $f(C) = \{mn|\ (m,n) \in \mathbb{N} \times \mathbb{N}: m \text{ is even or } n \text{ is even}\}$\\
Case 1: $m$ even, $b$ even\\
Same as part $b$, $mn = 4\mathbb{N}$.
% d
\item
% e
\item 
% f
\item 
\end{enumerate}
\end{prob}
\end{document}