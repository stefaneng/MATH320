\documentclass{article}

\usepackage[utf8]{inputenc}
\usepackage[top=1.5in,left=.9in,right=.9in,bottom=1in,headheight=1in]{geometry}
\usepackage{fancyhdr}
\usepackage{lastpage}
\usepackage{amsmath,amssymb,amsthm,mathrsfs}
\usepackage{mathrsfs}
\usepackage{multicol}
\usepackage{enumerate}



% Better looking empty set
\let\emptyset\varnothing

\newtheoremstyle{problem}{\topsep}{\topsep}%
{}%         Body font
{}%         Indent amount (empty = no indent, \parindent = para indent)
{\bfseries}% Thm head font
{\vspace{5pt}}%        Punctuation after thm head
{\newline}%     Space after thm head (\newline = linebreak)
{\thmname{#1}\thmnote{ #3}}%         Thm head spec

\theoremstyle{problem}
\newtheorem{prob}{Problem}

\theoremstyle{plain}
%\newtheorem{thm}{Theorem}
\newtheorem{lem}{Lemma}
\newtheorem{prop}{Proposition}
% No indent for whole page
\setlength\parindent{0pt}

\theoremstyle{remark}
\newtheorem{countex}{Counterexample}

\setlength{\columnsep}{1cm}
%%% Heading -- No need to edit %%%
\pagestyle{fancy}

\rhead{
  Stefan Eng\\
  Dr. Katherine Stevenson\\
  Math 320\\
  11/20/13
}
\lhead{
  Chapter 5\\
  67, 69, 70, 72, 75
}

\cfoot{Page\ \thepage\ of\ \pageref{LastPage}}

%%%

\renewcommand{\qedsymbol}{$\blacksquare$}
% For manual qed box
\def\bs{\hspace{\stretch1}\ensuremath\blacksquare}

\begin{document}

%%% Make the title %%%
\begin{center}
\textsc{\Large Foundations of Higher Mathematics}\\[.3cm]
\textsc{\Large Homework 11}
\end{center}
%%% End title %%%

%%% Start Assignment Here %%%
\begin{prob}[67] $f: \mathbb{R} \to \mathbb{R}$ such that $f(x) = 3x^2 + 2$.
\begin{enumerate}[a)]
\item $f([2,3])$\\
$f(2) = 3(2)^2 + 2 = 14$ and $f(3) = 3(3)^2 + 2 = 29$.
So, $f([2,3]) = [14,29]$.
\item $f^{-1}([55,307]) = \{ x \in \mathbb{R}|\ f(x) \in [55,307]\}$\\
\begin{multicols}{2}
\begin{align*}
  55 &= 3x^2 + 2\\
  53 &= 3x^2\\
  \frac{53}{3} &= x^2\\
  \pm \sqrt{\frac{53}{3}} &= x
\end{align*}
\vfill
\columnbreak
\begin{align*}
  307 &= 3x^2 + 2\\
  305 &= 3x^2\\
  \pm \sqrt{\frac{305}{3}} &= x
\end{align*}
\end{multicols}
So, $f^{-1}([55,307]) = \left [-\sqrt{\frac{53}{2}}, \sqrt{\frac{305}{5}}\  \right ] $
% c
\item $f^{-1}([1,2]) = \{x \in \mathbb{R}|\ f(x) \in [1,2]\}$\\
\begin{align*}
  1 \leq 3x^2 + 2 \leq 2\\
  -1 \leq 3x^2 \leq 0
\end{align*}
Since $3x^2 = -1$ has no solutions in $\mathbb{R}$, $f^{-1}([1,2]) = \{0\}$.
% d
\item \begin{align*}
    f^{-1}(f(3)) &= f^{-1}(3\cdot (3)^2 + 2)\\
    &= f^{-1}(29)\\
    29 &= 3x^2 + 2\\
    27 &= 3x^2\\
    x &= \pm 3
  \end{align*}  

% e
\item $f(\{-1, -2, -3\}) = \{3(-1)^2 + 2, 3(-2)^2 + 2, 3(-2)^2 + 2\} = \{5, 14, 29\}$

% f
\item $f(\{1, 2, 3\}) = \{5, 14, 29\}$, same work as last problem.
\end{enumerate}
\end{prob}

\begin{prob}[69]
\begin{enumerate}[a)]\ \\[-1cm]
% a
\item $f(A) = \{mn|\ (m,n) \in \mathbb{N} \times \mathbb{N}: m = 1 \text{ and } n \text{ is even}\}$\\
$= 1\cdot n$ where $n$ is even, so $f(A)$ is all even numbers.
% b
\item $f(B) = \{mn|\ (m,n) \in \mathbb{N} \times \mathbb{N}: m \text{ and } n \text{ are even}\}$\\
If $m$ and $n$ are even then, $m = 2j$ and $n = 2k$, so $mn = 4jk$. Thus, $f(B) = 4\mathbb{N}$.
% c
\item $f(C) = \{mn|\ (m,n) \in \mathbb{N} \times \mathbb{N}: m \text{ is even or } n \text{ is even}\}$\\
\textbf{Case 1}: $m$ even, $b$ even\\
Same as part $b$, $mn = 4\mathbb{N}$.\\
\textbf{Case 2}: $m$ even, $b$ odd\\
$2k \cdot 2j+1 = 2(2jk + k)$ so $mn$ is even.\\
\textbf{Case 3}: $m$ odd, $b$ even\\
$mn$ is even.\\

Since all $4\mathbb{N}$ are in $2\mathbb{N}$, $f(C) = 2\mathbb{N}$
% d
\item \begin{align*}
    f^{-1}(D) &= \{m,n \in \mathbb{N}|\ f(m,n) \in D\}\\
    &= \{m,n \in \mathbb{N}|\ f(m,n) \in \{x \in \mathbb{N}|\ x \text{ is odd}\}\\  
    &= \{m,n \in \mathbb{N}|\ m\cdot n \in \{x \in \mathbb{N}|\ x \text{ is odd}\}\\  
  \end{align*}
  $m\cdot n$ is only odd when $m$ is odd and $n$ is odd. ($m = 2k + 1$, $n = 2j + 1$, $mn = 4jk + 2j + 2k + 1 = 2(2jk + j + k) + 1)$. So, $f^{-1}(D) = (m,n)$ such that $m$ is odd and $n$ is odd.
  % e
\item \begin{align*}
    f^{-1}(E) &= \{m,n \in \mathbb{N}|\ f(m,n) \in \{n \in \mathbb{N}|\ x \text{ is even} \}\\
    &= \mathbb{N} \times \mathbb{N}
  \end{align*}
Because with even $m$ and odd $n$, $m\cdot n$, similarly for odd $m$ and even $n$. So, all pairs of $\mathbb{N}$ are in the set.
  % f
\item \begin{align*}
    f^{-1}(F) &= \{m,n \in \mathbb{N}|\ f(m,n) = 14\}\\
    &= \{ (1,14), (14,1), (2,7), (7,2) \}
  \end{align*}
\end{enumerate}
\end{prob}

\begin{prob}[70]
Let $f: \mathbb{R} \to \mathbb{R}$ be defined by $f(x) = \lfloor x \rfloor$, and let $A = [3,5] \cup (7,9) \cup (11,15)$.
\begin{itemize}
% a
\item $f(A) = \{f(a)|\ a \in A\} = [3,5] \cup [7,8] \cup [11,14]$
% b
\item $f^{-1}(A) = \{x \in \mathbb{R}|\ f(x) \in A\} = [3,6) \cup [8,9) \cup [12,15)$.
% c
\item $f(f^{-1}(A)) = f([3,6) \cup [8,9) \cup [12,15)) = [3,5] \cup \{8\} \cup [12,14)$
% d
\item $f^{-1}(f(A)) = \{x \in \mathbb{R}|\ f(x) \in f(A)\} = [3,6] \cup [7,9) \cup [11,15)$
\end{itemize}
\end{prob}

% 72
\begin{prob}[72]\ \\[-1cm]
\begin{proof}
  Assume $A,B \in \mathcal{P}(X)$ such that $A \not = B$. 
  That is, there exists an $x \in A$ and $x \not \in B$, or $x \not \in A$ and $x \in B$.
  Assume $x \in A$ and $a \not \in B$. (Identical argument follows for $x \not \in A$, $x \in B$)
  Since, $x \in A$, $x \not \in X - A$. 
  It follows that since $x \in A$ and $A \subseteq X$, that $x \in X$. 
  Now take $C_X(B) = X - B$. 
  Since $x \in X$, and $x \not \in B$, $x \in X - B$. 
  Since there exists $x \in X - A$ and $x \not \in X - B$, then $X - A \not = X - B$ and thus, $C_x(A) \not = C_x(B)$. Therefore, we have shown that $C_x$ is injective.\\
\end{proof}

\begin{align*}
  C_X^{-1}(A) &= \{B \in \mathcal{P}(X)|\ C_x(B) = A\}\\
  &= \{B \in \mathcal{P}(X)|\ X - B = A \}
\end{align*}
So $C^{-1}_X$ is just the set that is the complement of $A$.
\end{prob}

% 75
\begin{prob}[75]
($\Leftarrow$) Assume $f(A) \cap f(B) = f(A \cap B)$. Let $x_1, x_2 \in X$ such that $f(x_1) = f(x_2)$. Want to show that $x_1 = x_2$. Since $A$ and $B$ are sets, represent $x_1$ and $x_2$ as one element sets, $\{x_1\}$ and $\{x_2\}$. It follows that $f(\{x_1\}) \cap f(\{x_2\}) = f(\{x_1\} \cap \{x_2\})$. Since $f(x_1) = f(x_2)$, and the intersection of two identical sets is itself, $f(\{x_1\}) = f(\{x_1\} \cap \{x_2\})$. It follows that:
$$
  f(\{x_1\}) = \{f(x)|\ x \in \{x_1\} \text{ and } x \in \{x_2\}\}  
$$
It follows that $x_1 = x_2$. Therefore, $f$ is injective.\\
($\Rightarrow$) 
Assume $f$ is $injective$. 
We want to show, $f(A) \cap f(B) = f(A \cap B)$.\\
($\subseteq$)
Assume an arbitrary $y \in f(A) \cap f(B)$. 
It follows that $y \in f(A)$ and $y \in f(B)$. 
By definition, $y \in \{f(a)|\ a \in A\}$ and $y \in \{f(b)|\ b \in B\}$. 
%Since $A$ and $B$ are subsets of $X$, $a \in X$ and $b \in X$.
We can see that $y \in Im(f)$ because $A$ and $B$ are subsets of $X$.
It follows that there is an unique element, call it $x$, such that $f(x) = y$.
Since $y \in f(A)$ and $y \in f(B)$, this unique $f(x) \in f(A)$ and $f(x) \in f(B)$.
Since $x$ is unique, $x \in A$ and $x \in B$.
So, $x \in A \cap B$.
It follows that $f(x) \in f(A \cap B)$.
Finally, $y \in f(A \cap B)$.\\
($\supseteq$) Assume an arbitrary $y \in f(A \cap B)$. Since $y \in Im(f)$, there is a unique element, $x$, such that $f(x) = y$. So, $x \in A \cap B$. It follows that $x \in A$, and $x \in B$. Thus, $f(x) \in f(A)$ and $f(x) \in f(B)$. Therefore, since $f(x) = y$, $y \in f(A) \cap f(B)$.\\
Proving both subsets, we can conclude that $f(A) \cap f(B) = f(A \cap B)$.
\end{prob}
\end{document}