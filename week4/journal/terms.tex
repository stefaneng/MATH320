\documentclass{article}

\usepackage[utf8]{inputenc}
\usepackage[top=1.5in,left=1in,right=1in,bottom=1.5in,headheight=1in]{geometry}
\usepackage{amsmath,amsthm,amsfonts}

\newtheorem{thm}{Thereom}

\theoremstyle{definition}
\newtheorem{defn}{Definition}[section]
\begin{document}
\section{Introduction}
These are various terms to hopefully help write a good journal. More of a rough draft of what the journal will look like.

\section{Problem}
\begin{thm}
If a \textbf{planar graph} has minimum \textbf{degree} 5, then it either has an \textbf{edge} with endpoints both of degree 5 or one with endpoints of degrees 5 and 6.
\end{thm}

\section{Terms}

\begin{defn}[Planar Graph]
Graph that can be drawn where no edges cross each other.
\end{defn}

\begin{defn}[Triangulated]
A planer graph $G$ is said to be triangulated if the addition of any edge to $G$ results in a nonplanar graph.
\end{defn}

\begin{defn}[Degree]
The degree of a vertex is number of edges ``connected'' with loops counting twice.
\end{defn}

\begin{defn}[Maximum/Minimum Degree]
  The max/min of a graph G is denoted by $\Delta(G)$ for the max degree of the graph, and $\delta(G)$ for the min degree of the graph.
\end{defn}



\end{document}