\documentclass{article}

\usepackage[utf8]{inputenc}
\usepackage[top=1.5in,left=1in,right=1in,bottom=1.5in,headheight=1in]{geometry}
\usepackage{fancyhdr}
\usepackage{lastpage}
\usepackage{amsmath,amssymb,amsthm}


% Better looking empty set
\let\emptyset\varnothing

\newtheoremstyle{problem}{\topsep}{\topsep}%
{}%         Body font
{}%         Indent amount (empty = no indent, \parindent = para indent)
{\bfseries}% Thm head font
{\vspace{5pt}}%        Punctuation after thm head
{\newline}%     Space after thm head (\newline = linebreak)
{\thmname{#1}\thmnote{ #3}}%         Thm head spec

\theoremstyle{problem}
\newtheorem{prob}{Problem}

%\theoremstyle{plain}
%\newtheorem{thm}{Theorem}

% No indent for whole page
\setlength\parindent{0pt}

%%% Heading -- No need to edit %%%
\pagestyle{fancy}

\rhead{
  Stefan Eng\\
  Dr. Katherine Stevenson\\
  Math 320\\
  9/25/13
}
\lhead{
  (3.1) 1d, 1j, 1m, 4, 5, 8, 11\\
  (3.2-3.3) 20, 25, 28, 35, 36, 39, 40, 53
}

\cfoot{Page\ \thepage\ of\ \pageref{LastPage}}
%%%

\renewcommand{\qedsymbol}{$\blacksquare$}
% For manual qed box
\def\bs{\hspace{\stretch1}\ensuremath\blacksquare}

\begin{document}

%%% Make the title %%%
\begin{center}
\textsc{\Large Foundations of Higher Mathematics}\\[.3cm]
\textsc{\Large Homework 4}
\end{center}
%%% End title %%%

%%% Start Assignment Here %%%

\section*{Section 3.1}
\begin{prob}[1d]
  %\begin{thm}
  $5 + 10 + 15 + \ldots + 5n = \frac{5n(n + 1)}{2}$
  %\end{thm}
  \begin{proof}\ 
    \begin{enumerate}
      \item S = \{$n \in \mathbb{N}$ $|\ 5 + 10 + 15 + \ldots + 5n = \frac{5n(n + 1)}{2}$\}
      \item (Basis) n = 1 implies, $5(1) = \frac{5(1)(1 + 1)}{2} = 5$, thus $1 \in S$
      \item (Induction Hypothesis) Assume $n \in S$. Thus, $5 + 10 + 15 + \ldots + 5n = \frac{5n(n + 1)}{2}$. Then we want to show $n + 1 \in S$.      
    \end{enumerate}
    \begin{align*}
      [5 + 10 + 15 + \ldots + 5n] + 5(n + 1) &= \frac{5n(n + 1)}{2} + 5(n + 1) \tag*{(Induction hypothesis)}\\
      &= 5(n + 1)(\frac{n}{2} + 1)\\
      &= 5(n + 1)(\frac{n + 2}{2})\\
      &= \frac{5(n + 1)[(n+1) + 1]}{2}
    \end{align*}
    Since we can clearly see that $n + 1 \in S$. Therefore, by the Principle of Mathematical Induction, $S = \mathbb{N}$.
  \end{proof}
  \end{prob}
  \begin{prob}[1j]
    $7^n - 2^n$ is divisible by 5.
    \begin{proof}\  
      \begin{enumerate}
      \item S = \{$n \in \mathbb{N}$ $|\ 5 | (7^n - 2^n)$\}
      \item (Basis) $n = 1$ implies, $5 | (7^1 - 2^1)$. Thus, $5(1) = 5$, so $1 \in S$.
      \item (Induction Hypothesis) Assume $m \in S$. Then, $5 | (7^n - 2^n)$. Want to show that $m + 1 \in S$.
      \end{enumerate}
      \begin{align*}
        7^{m+1} - 2^{m+1} &= 7^{m+1} -7(2^m) + 7(2^m) - 2^{m+1}\\
        &= 7\underbrace{(7^m - 2^m)}_{\text{(Ind. Hyp)}} + 2^m(7 - 2)\\
        &= 7(5k) + 2^m(5)\\
        &= 5(7k + 2^m)
      \end{align*}
      Since, $7k + 2^m \in \mathbb{Z}$, we have showed that $5 | (7^{m+1} - 2^{m+1})$. Thus proving that $m + 1 \in S$. Therefore, by the Principle of Mathematical Induction, $S = \mathbb{N}$.
    \end{proof}
  \end{prob}
 
  \section*{Section 3.2}

\section*{Section 3.3}

\end{document}