\documentclass{article}

\usepackage[utf8]{inputenc}
\usepackage[top=1.5in,left=1in,right=1in,bottom=1.5in,headheight=1in]{geometry}
\usepackage{fancyhdr}
\usepackage{lastpage}
\usepackage{amsmath,amssymb,amsthm}
\usepackage{multicol}
\usepackage{enumerate}


% Better looking empty set
\let\emptyset\varnothing

\newtheoremstyle{problem}{\topsep}{\topsep}%
{}%         Body font
{}%         Indent amount (empty = no indent, \parindent = para indent)
{\bfseries}% Thm head font
{\vspace{5pt}}%        Punctuation after thm head
{\newline}%     Space after thm head (\newline = linebreak)
{\thmname{#1}\thmnote{ #3}}%         Thm head spec

\theoremstyle{problem}
\newtheorem{prob}{Problem}

\theoremstyle{plain}
%\newtheorem{thm}{Theorem}
\newtheorem{lem}{Lemma}
% No indent for whole page
\setlength\parindent{0pt}

\setlength{\columnsep}{1cm}
%%% Heading -- No need to edit %%%
\pagestyle{fancy}

\rhead{
  Stefan Eng\\
  Dr. Katherine Stevenson\\
  Math 320\\
  9/25/13
}
\lhead{
  (3.1) 1d, 1j, 1m, 4, 5, 8, 11\\
  (3.2) 20, 25, 28, 35, 36
}

\cfoot{Page\ \thepage\ of\ \pageref{LastPage}}
%%%

\renewcommand{\qedsymbol}{$\blacksquare$}
% For manual qed box
\def\bs{\hspace{\stretch1}\ensuremath\blacksquare}

\begin{document}

%%% Make the title %%%
\begin{center}
\textsc{\Large Foundations of Higher Mathematics}\\[.3cm]
\textsc{\Large Homework 4}
\end{center}
%%% End title %%%

%%% Start Assignment Here %%%

\section*{Section 3.1}
\begin{prob}[1d]
  %\begin{thm}
  $5 + 10 + 15 + \ldots + 5n = \frac{5n(n + 1)}{2}$
  %\end{thm}
  \begin{proof}
    Let S = \{$n \in \mathbb{N}$ $|\ 5 + 10 + 15 + \ldots + 5n = \frac{5n(n + 1)}{2}$\}.
    Then n = 1 implies, $5(1) = \frac{5(1)(1 + 1)}{2} = 5$, thus $1 \in S$.
    Assume $n \in S$. Thus, $5 + 10 + 15 + \ldots + 5n = \frac{5n(n + 1)}{2}$ (Induction Hypothesis). Then we want to show $n + 1 \in S$.      
    \begin{align*}
      [5 + 10 + 15 + \ldots + 5n] + 5(n + 1) &= \frac{5n(n + 1)}{2} + 5(n + 1) \tag*{(Induction hypothesis)}\\
      &= 5(n + 1)(\frac{n}{2} + 1)\\
      &= 5(n + 1)(\frac{n + 2}{2})\\
      &= \frac{5(n + 1)[(n+1) + 1]}{2}
    \end{align*}
    Since we can clearly see that $n + 1 \in S$. Therefore, by the Principle of Mathematical Induction, $S = \mathbb{N}$.
  \end{proof}
  \end{prob}
  \begin{prob}[1j]
    $7^n - 2^n$ is divisible by 5.
    \begin{proof}(Induction)
      \begin{enumerate}
      \item S = \{$n \in \mathbb{N}$ $|\ 5 | (7^n - 2^n)$\}
      \item (Basis) $n = 1$ implies, $5 | (7^1 - 2^1)$. Thus, $5(1) = 5$, so $1 \in S$.
      \item (Induction Hypothesis) Assume $m \in S$. Then, $5 | (7^m - 2^m)$. Want to show that $m + 1 \in S$.
      \end{enumerate}
      \begin{align*}
        7^{m+1} - 2^{m+1} &= 7^{m+1} -7(2^m) + 7(2^m) - 2^{m+1}\\
        &= 7\underbrace{(7^m - 2^m)}_{\text{(Ind. Hyp)}} + 2^m(7 - 2)\\
        &= 7(5k) + 2^m(5)\\
        &= 5(7k + 2^m)
      \end{align*}
      Since, $7k + 2^m \in \mathbb{Z}$, we have showed that $5 | (7^{m+1} - 2^{m+1})$. Thus proving that $m + 1 \in S$. Therefore, by the Principle of Mathematical Induction, $S = \mathbb{N}$.
    \end{proof}
  \end{prob}
 
  \begin{prob}[1m]
    $4^n - 1$ is divisible by 3.
    \begin{proof}(Induction) 
      \begin{enumerate}
      \item S = \{$n \in \mathbb{N}$ $|\ 3 | (4^n - 1)$\}
      \item (Basis) $n = 1$ implies, $3 | (4^1 - 1)$. Thus, $3(1) = 3$, so $1 \in S$.
      \item (Induction Hypothesis) Assume $m \in S$. Then, $3 | (4^m - 1)$. Want to show that $m + 1 \in S$.
      \end{enumerate}
      \begin{align*}
        4^{m+1} - 1 &= 4^{m+1} - 4 + 4 - 1\\
        &= 4\underbrace{(4^m - 1)}_{\text{(Ind. Hyp)}} + 3\\
        &= 4(3k) + 3\\
        &= 3(4k + 1)
      \end{align*}
      Since $4k + 1 \in \mathbb{Z}$, $3 | (4^{m+1} - 1)$. Therefore, $m+1 \in S$. So by the Principle of Mathematical Induction, $S = \mathbb{N}$.
    \end{proof}
  \end{prob}

  \begin{prob}[4]
    Blah
  \end{prob}

  \begin{prob}[5]
    The product of any three consecutive natural numbers is divisible by 6.
    \begin{multicols}{2}
    \begin{proof}
      Let $S = \{n \in \mathbb{N} |\ 6 | n(n+1)(n+2)\}$. Then if $n = 1$, we see that $6|1(1+1)(1+2) = 6|6$, 1 is clearly in S. Assume that $k \in S$ such that, $6|k(k+1)(k+2)$. We want to show that $k+1 \in S$.
      \begin{align*}
        (k+1)&[(k+1)+1][(k+1)+2] \\
        &= k+1(k+2)(k+3)\\
        &= \underbrace{k(k+1)(k+2)}_{Ind. Hyp.} + 3\underbrace{(k+1)(k+2)}_{Lemma}\\
        &= 6q + 3(2r)\\
        &= 6(q + r)        
      \end{align*}
      Since $q + r \in \mathbb{Z}$, we can clearly see that it is divisible by 6. Thus proving that $k + 1 \in S$ and so by the Principle of Mathematical Induction, $S=\mathbb{N}$.
    \end{proof}
    \vfill
    \columnbreak
    \begin{lem}
      The product of two consecutive natural number is divisible by 2
      \begin{proof}
        We will prove the two cases of starting with an odd number, and starting with and even.
        \begin{enumerate}[{Case} 1)]
        \item Assume we start with an even number. Let's call that number 2h. Then the next consecutive number is 2h+1. The product of these numbers is as follows:
          \begin{align*}
            2h(2h+1) &= 4h^2 + 2h\\
            &= 2(2h^2 + h)
          \end{align*}
          Since $2h^2 + h \in \mathbb{Z}$, we have shown that 2 divides it.
        \item Assume now that we start with an odd number. The proof is similar:
          \begin{align*}
            2h+1(2h+2) &= 4k^2 + 4k + 2k +2\\
            &= 2(2k^2 + 3k + 1)
          \end{align*} 
          Thus, 2 divides the product of two consecutive numbers (starting with an odd number).
        \end{enumerate}
        Therefore, since we have proved both the odd and even starting case, the product of two consecutive numbers is divisible by 2.
      \end{proof}      
    \end{lem}
  \end{multicols}
  \end{prob}

  \begin{prob}[8]
    Blah Blah
  \end{prob}

  \begin{prob}[11]
    Blah
  \end{prob}
  \section*{Section 3.2}
  \begin{prob}[20]
    Prove that for each natural number $n \geq 6$, $(n + 1)^2 \leq 2^n$.
    \begin{multicols}{2}      

      \begin{proof}
        Let $S = \{n \in \mathbb{N} |\ n \geq 6 \wedge (n + 1)^2 \leq 2^n\}$. If $n = 6$, we can see that $(6+1)^2 = 49 \leq 64 = 2^6$. So clearly, $6 \in S$. Assume $m \in S$. Thus, $(m + 1)^2 \leq 2^m$. We want to show that $m + 1 \in S$.
        \begin{align*}
          [(m+1) + 1]^2 &= m^2 + 4m + 4\\
          &= \underbrace{(m^2 + 2m + 1)}_{Ind. Hyp.} + (2m + 3)\\
          &\leq 2^m + (2m + 3)\\
          &\leq 2^m + 2^m \tag*{(Lemma)}\\
          &=2^{m+1}
        \end{align*}
        So clearly, $m + 1 \in S$. Therefore, by the Extended Principle of Mathematical Induction, $$\{n \in \mathbb{N} : n \geq 6\} \subseteq S$$
      \end{proof}     
      \vfill
      \columnbreak
      \begin{lem}
        For all $k \geq 6$, $2k + 3 \leq 2^k$
        \begin{proof}
          Let S = $\{k \in \mathbb{N} |\ k \geq 6 \wedge 2k + 3 \leq 2^k\}$. When $n = 6$, $2(6) + 3 = 15 \leq 64 = 2^6$. It follows that $6 \in S$. Assume $j \in S$. Thus, $2j + 3 \leq 2^j$. Want to show that $j + 1 \in S$.
          \begin{align*}
            2^{j+1} &= 2\underbrace{(2^j)}_{Ind. Hyp.}\\
            &\geq 2(2j+3)\\
            &= 4j + 6\\
            &\geq 4j + 6 - (2j + 1)\\
            &= 2j+5\\
            &= 2(j+1) + 3
          \end{align*}
          We can clearly see that $j + 1 \in S$. Therefore, by the Extended Priciple of Mathematical Induction, $\{k \in N : k \geq 6\} \subseteq S$.
        \end{proof}
      \end{lem}
    \end{multicols}
  \end{prob}
  \begin{prob}[25]
    Blah
  \end{prob}
  \begin{prob}[28]
    Blah
  \end{prob}
  \begin{prob}[35]
    Blah
  \end{prob}
  \begin{prob}[36]
    Blah
  \end{prob}
\end{document}