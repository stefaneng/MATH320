\documentclass{article}

\usepackage[utf8]{inputenc}
\usepackage[top=1.5in,left=1in,right=1in,bottom=1.5in,headheight=1in]{geometry}
\usepackage{fancyhdr}
\usepackage{amsmath,amsthm,amsfonts}
%\usepackage{indentfirst}
\usepackage{setspace}

\theoremstyle{defintion}
\newtheorem*{defn}{Definition}
\newtheorem*{nota}{Notation}

\onehalfspacing

%%%%%%%%%%%%%%%%%%%%%%%%%%%%%
%%% Information for Title %%%
%%%%%%%%%%%%%%%%%%%%%%%%%%%%%

%%% Fill this in with your own information

\newcommand{\name}{
% Enter your name here
Stefan Eng
}

\newcommand{\professor}{
% Professors name goes here
Dr. Katherine Stevenson
}

\newcommand{\classcode}{
% Class name "code" goes here. e.g, Math320
Math 320
}

\newcommand{\classname}{
% Class name goes here
Foundations of Higher Mathematics
}

\newcommand{\duedate}{
% Due date goes here
9/9/13
}

\newcommand{\journal}{
% Enter journal number
2
}

%%%%%%%%%%%%%%%%%%%%%%%%%%%%%


%%% Heading -- No need to edit %%%
\pagestyle{fancy}
\rhead{\name \\ \professor \\ \classcode \\ \duedate}
%%%

% For no qed symbol
\renewcommand{\qedsymbol}{}
% Good reference for math commands web.ift.uib.no/Teori/KURS/WRK/TeX/symALL.html

\begin{document}
\pagenumbering{gobble}

%%% Make the title %%%
\begin{center}
\textsc{\Large \classname}\\[.3cm]
\textsc{\Large Journal \journal}\\[1cm]
\end{center}
%%% End title %%%

%%% Start Journal Here Here %%%
\section*{Overview}
This week we looked at some basic set theory terms and symbolic notation. Some of the terms we discussed were subsets and powersets. We briefly touched on cardinality and countability. When we discussed that $|\mathbb{N}| = |\mathbb{Q}|$ there was quite a commotion in class. Since we just had one day of class there was not too much to write about so I thought I would recap some of the definitions and notation we used this week.

\section*{Notation and Definitions}
\begin{defn}[set]
A \textbf{set} is a well defined collection of objects.
\end{defn}

\begin{nota}
\indent If we have an object $a$ and a set $A$ and $a$ is an element of $A$ then we write $$a \in A$$
\end{nota}

\begin{defn}[subset]
Given sets $A$ and $B$, $B$ is a \textbf{subset} of $A$ if every element of $B$ is in $A$.
\end{defn}

\begin{nota}
If $A$ is a \textbf{subset} of $B$ we write, $$A \subseteq B$$ which is equivalent to, $$\forall x(x \in A \to x \in B)$$
\end{nota}

\begin{defn}[power set]
The \textbf{power set} of $A$ is the set of every subset of $A$.
\end{defn}

\begin{nota}
The power set of $A$ is defined as, $$\mathcal{P}{(A)}$$
which can be expressed in symbols as,
$$\forall b \exists a \forall x(x \in a \leftrightarrow x \subseteq b)$$
or expressed in set builder notation:
$$ \mathcal{P}{(A)} = \{ x\ |\ x \subseteq A \} $$
\end{nota}

\begin{defn}[cardinality]
The number of elements in a set is called its \textbf{cardinality}.
\end{defn}

\begin{nota}
Given a set $A$ the cardinality of $A$ is,
$$\left\vert{A}\right\vert $$
\end{nota}

\end{document}

%%% End assignment %%%