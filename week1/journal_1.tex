\documentclass{article}

\usepackage[utf8]{inputenc}
\usepackage[top=1.5in,left=1in,right=1in,bottom=1.5in,headheight=1in]{geometry}
\usepackage{fancyhdr}
\usepackage{amsmath,amsthm,amsfonts}
\usepackage{indentfirst}

%%%%%%%%%%%%%%%%%%%%%%%%%%%%%
%%% Information for Title %%%
%%%%%%%%%%%%%%%%%%%%%%%%%%%%%

%%% Fill this in with your own information

\newcommand{\name}{
% Enter your name here
Stefan Eng
}

\newcommand{\professor}{
% Professors name goes here
Dr. Katherine Stevenson
}

\newcommand{\classcode}{
% Class name "code" goes here. e.g, Math320
Math 320
}

\newcommand{\classname}{
% Class name goes here
Foundations of Higher Mathematics
}

\newcommand{\duedate}{
% Due date goes here
9/2/13
}

\newcommand{\journal}{
% Enter journal number
1
}

%%%%%%%%%%%%%%%%%%%%%%%%%%%%%


%%% Heading -- No need to edit %%%
\pagestyle{fancy}
\rhead{\name \\ \professor \\ \classcode \\ \duedate}
%%%

% For no qed symbol
\renewcommand{\qedsymbol}{}
% Good reference for math commands web.ift.uib.no/Teori/KURS/WRK/TeX/symALL.html

\begin{document}
\pagenumbering{gobble}

%%% Make the title %%%
\begin{center}
\textsc{\Large \classname}\\[.3cm]
\textsc{\Large Journal \journal}\\[1cm]
\end{center}
%%% End title %%%

%%% Start Journal Here Here %%%
\section*{Overview}
This week in Foundations of Higher Mathematics we learned about the logic and language of proofs. In class we analysed sentences and determined whether or not they were propositions. If they were propositions, we determined if they were true of false. We also looked at some of the logical statements such as \textit{and}, \textit{or}, \textit{not}, \textit{if-then} and \textit{if-and-only-if}. We built truth tables for all of the logical operations we looked at. Also regarding propositions, we looked at the contrapositive and converse of an \textit{if} statement. Then we investigated expressions and tautologies and how to determine if two statements are equivalent. We finished up the week with quantifiers. We learned the various ways to express the universal and existential quantifier in English. An interesting part of this was investigating the language that is use speech that hides the use of quantifiers.

\section*{Example}
One part of this week that particularly stood out as was our first proofs in the notes. The three "warmup" lemmas and proofs from Wednesday this week were fun so I thought I would show the proofs for those.

\newtheorem{lem}{Lemma}

\begin{lem}
Let m and n be odd integers, then mn is an odd integer.
\end{lem}

\begin{proof}
Since $m$ is odd, there exists a $k$ such that, $m = 2k + 1$. Similarly for $n$, $n = 2l + 1$. It follow that 
\[
    \arraycolsep=1.5pt
    \begin{array}[t]{rl}
    mn     &= (2k+1)(2l+1) \\
           &= 4kl+2k+2l+1\\
        &= 2(2kl+k+l)+1\\      
    \end{array}
\]
Hence, $mn = 2b+1$ where $b = (2kl+k+l) \in \mathbb{Z}$. Thus, $mn$ is an odd integer.
\end{proof}

\begin{lem}
If b is odd then $b^3$ is odd.
\end{lem}

\begin{proof}
Suppose $b$ is odd. Then by lemma 1, $(b)(b) = b^2$ is odd. Since $b$ is odd and $b^2$ is odd, it follows again from lemma 1 that $b^3$ is odd.
\end{proof}

\begin{lem}
If $b^3$ is even then b is even
\end{lem}

\begin{proof}
Looking at the contrapositive, we see that we proved this in Lemma 2.  
\end{proof}

\end{document}

%%% End assignment %%%