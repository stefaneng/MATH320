\documentclass{article}

\usepackage[utf8]{inputenc}
\usepackage[top=1.5in,left=1in,right=1in,bottom=1.5in,headheight=1in]{geometry}
\usepackage{fancyhdr}
\usepackage{lastpage}
\usepackage{amsmath}

\newcommand{\ndiv}{\hspace{-4pt}\not|\hspace{2pt}} 

%%%%%%%%%%%%%%%%%%%%%%%%%%%%%
%%% Information for Title %%%
%%%%%%%%%%%%%%%%%%%%%%%%%%%%%

%%% Fill this in with your own information

\newcommand{\name}{
% Enter your name here
Stefan Eng
}

\newcommand{\professor}{
% Professors name goes here
Dr. Katherine Stevenson
}

\newcommand{\classcode}{
% Class name "code" goes here. e.g, Math320
Math 320
}

\newcommand{\classname}{
% Class name goes here
Foundations of Higher Mathematics
}

\newcommand{\duedate}{
% Due date goes here
9/4/13
}

\newcommand{\chSec}{
% Enter in chapter sections
1.1-1.3
}
    
\newcommand{\problems}{
% Enter in the problem numbers we are working on
6, 13, 27, 39, 41, 42
}

\newcommand{\assignment}{
% Enter assignment number
1
}

%%%%%%%%%%%%%%%%%%%%%%%%%%%%%


%%% Heading -- No need to edit %%%
\pagestyle{fancy}
\rhead{\name \\ \professor \\ \classcode \\ \duedate}
\lhead{Section:\chSec\ \\ Problems:\problems}
\cfoot{Page\ \thepage\ of\ \pageref{LastPage}}
%%%


\begin{document}

%%% Make the title %%%
\begin{center}
\textsc{\Large \classname}\\[.3cm]
\textsc{\Large Homework \assignment}
\end{center}
%%% End title %%%

%%% Start Assignment Here %%%
\section*{Problem 6}
Assume that "Joe is a girl" is false and "Joe is ten years old" is false as well. So this means that Joe is a boy and Joe is not ten years old. Thus the true statements are,
\begin{enumerate}
\item[b)] If Joe is ten years old, then Joe is a girl.
\item[c)] Joe is ten years old if and only if Joe is a girl.
\item[d)] Joe is not a ten year-old girl.
\end{enumerate}

\section*{Problem 13}
\begin{enumerate}
    \item[a)] The contrapositive of, 
    $if\ 3 > 1,\ then\ 5 > 1$ is, $if\ 5\ \leq 1,\ then\ 3 \leq 1$, 
    so the converse of that is: $$if\ 3 \leq 1,\ then\ 5 \leq 1$$
    
    \item[b)] The converse of, 
    $if\ 3 > 1,\ then\ 5 > 1$ is, $if\ 5 > 1,\ then\ 3 > 1$,
    so the contrapositive is:
    $$if\ 3 \leq 1,\ then\ 5 \leq 1$$
\end{enumerate}

\section*{Problem 27}
Suppose that each of these statements is true
\begin{itemize}
\item[] John is smart.
\item[] John or Mary is ten years old.
\item[] If Mary is ten years old, then John is not smart.
\end{itemize}
then\ldots
\begin{enumerate}
\item[a)] Mary is ten years old is \textbf{false}.\\
By the contrapositive of the third proposition, if John is smart, then Mary is not ten years old. Since John is smart, we can conclude that Mary is not ten years old.

\item[b)] John is ten years old is \textbf{true}. \\
We can conclude that since John or Mary is ten years old, and Mary is not ten, from part a, John must be ten years old.

\item[c)] Either John or Mary is not ten years old is \textbf{true}.\\
Since we concluded that Mary is not ten years old from part a, this statement is true.
\end{enumerate}

\section*{Problem 39}
If $x$ is an even integer or $x > 17$, then $x$ is a multiple of $4$ and $x\leq5$.

\begin{description}
\item[Negative:] $x$ is an even integer or $x > 17$, and $x$ is not a multiple of $4$ or $x > 5$.
\item[Contrapositive:] If $x$ is not a multiple of $4$ or $x > 5$, then $x$ is an odd integer and $x \leq 17$.
\end{description}

\section*{Problem 41}
Negate the following statements:
\begin{enumerate}
\item[a)] All cows eat grass.\\[.2cm]
No cows eat grass.
\item[b)] There is a horse that does not eat grass.\\[.2cm]
All horses eat grass.
\item[c)] There is a car that is blue and weighs less that 4,000 pounds.\\[.2cm]
All cars are not blue or weigh at least 4,000 pounds.
\item[d)] Every math book is either blue or hard to read.\\[.2cm]
There is a math book that isn't blue and easy to read.
\item[e)] Some cows are spotted.\\[.2cm]
All cows are spotless.
\item[f)] No car has 15 cylinders.\\[.2cm]
There is a car with 15 cylinders.
\item[g)] Some cars are old but are still in good running condition.\\[.2cm]
All cars are new or in bad running condition.
\end{enumerate}

\section*{Problem 42}
For an integer x to have \textbf{property P} symbolically it means,
$$
\forall x \forall y (x \vert ab \to x \vert a \vee x \vert b)
$$
So for an integer x to \textbf{not} have \textbf{property P},
$$
\neg (\forall x \forall y (x \vert ab \to x \vert a \vee x \vert b)) \Leftrightarrow
$$
$$
\exists x \exists y (x \vert ab \wedge x \ndiv a \wedge x \ndiv b)
$$
which means, \textit{For an integer x to not have \textbf{property P}, there are integers a and b such that x divides ab, x does not divides a and x does not divides b}.
\end{document}
%%% End assignment %%%
