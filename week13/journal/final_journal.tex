\documentclass{article}

\usepackage[utf8]{inputenc}
\usepackage[top=1.5in,left=1in,right=1in,bottom=1.5in,headheight=1in]{geometry}
\usepackage{fancyhdr}
\usepackage{amsmath,amsthm,amsfonts,amssymb}
\usepackage{setspace}
\usepackage{multicol}

\onehalfspacing

% \theoremstyle{theorem}
\newtheorem*{thm}{Theorem}
\newtheorem*{lemma}{Lemma}
\renewcommand{\qedsymbol}{$\blacksquare$}

%%% Heading -- No need to edit %%%
\pagestyle{fancy}
\rhead{
Stefan Eng \\ 
Dr. Katherine Stevenson \\ 
Math320 \\ 
12/2/13
}
%%%

\begin{document}
\pagenumbering{gobble}

%%% Make the title %%%
\begin{center}
\textsc{\Large Foundations of Higher Mathematics}\\[.3cm]
\textsc{\Large Final Journal}\\[1cm]
\end{center}
%%% End title %%%

%%% Start Journal Here %%%

% In this journal please reflect on: 
% 1) whether or not you see that you have progressed as a mathematician, 
% 2) some math that you have seen this term (in any class) that made things "click" for you 
%    or that brought together several different areas of your studies.
% * Your journal should be at least a page long.  
% * It should include both prose and math symbols. 
% * The grammar and spelling should be perfect. 
% * Make this your best effort. 
% * This journal grade will not be dropped (the lowest of the other 7 journal entries will be dropped).

%Before I started this class I was was not a mathematician.
%I may have had an interest in mathematics, but nothing like how I ended this semester. 
%Coming from a Computer Science background, I thought mathematics would be a good supplement to my computer knowledge.

%Almost everything in this class made things ``click'' for me.
% This class made me feel like a mathematician.
I started out this semester a computer scientist.
I almost felt like an impostor, trying to blend in with the math majors.
This changed very quickly and it didn't take long before I realized how much I enjoyed math.
No other class has made this big of difference in my education.


What really made a huge difference was the journals we had to write.
I wrote my first \LaTeX  paper a week or two before the class started so I thought it would be a nice way to get some more practice with it.
While I definitely got much better at my typesetting, I feel that exploring a new topic every week helped me find a new passion for math.
Reading \textit{One, Two, Three,... Infinity}, had a big on the way I looked at math.
Using this book as my base for many journals I explored deeper some of the subjects discussed.
I read many Combinatorics papers and explored some of Georg Cantor's work on cardinality.
% This really helped spark an interest in mathematics beyond 
Every week I got to look forward to reading about a new topic to try to build on what I had done the previous weeks.
One of the more interesting theorems I came across was the \textit{Cantor–Bernstein–Schroeder theorem}:
\begin{thm}
If there exist injective functions $f : A \to B$ and $g : B \to A$ between A and B, then there exists a bijective function $h : A \to B$
\end{thm}
I first saw this theorem when I wrote a journal on cardinality, I had no idea what this theorem meant.
This is one of the many examples I saw earlier in the semester that now make much more sense.
I kept running into parts of my journals that I would try to grasp conceptually first, and then see the formal definitions in class.
I love approaching math from this point of view, where a intuitive understanding of theorems and definitions are investigated before formal definitions are given.

All of the topics I wrote about in my journals were discrete topics.
I felt comfortable with integers and rational numbers.
Calculus was for engineers; I was a computer scientist.
This mindset did not make it through this course.
As we progressed into more continuous topics things started to click in a different way.
These last few weeks on analysis made me realize how much I have learned this semester.
Last summer I picked up a cheap copy of ``baby Rudin'' and was in shock at how hard it was.
I struggled through maybe one or two of the theorems and then gave up, putting it away in my bookshelf.
When we started our work on analysis, I took another look at it and to my surprise I was actually able to do some of the problems.
Analysis may not be my favorite subject, but it was a huge leap to be able to understand (some of) ``Rudin's'' proofs.
Tackling a subject that has previously 
By the time I get to analysis I am sure that playing around with Rudin beforehand will make the class much easier.



To conclude, this class has more than anything else given me the confidence to pursue math. 
It really sparked an interest in mathematics beyond that of supplementing computer science.
Thanks for all of your supporting comments along that was that made the journals more exciting to write.
I feel like I am (on my way to becoming) a mathematician.


\begin{quote}

\end{quote}



\end{document}

%%% End assignment %%%
