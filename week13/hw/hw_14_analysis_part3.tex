\documentclass{article}

\usepackage[utf8]{inputenc}
\usepackage[top=1.5in,left=.9in,right=.9in,bottom=1in,headheight=1in]{geometry}
\usepackage{fancyhdr}
\usepackage{lastpage}
\usepackage{amsmath,amssymb,amsthm,mathrsfs}
\usepackage{mathrsfs}
\usepackage{multicol}
\usepackage{enumerate}

\usepackage{setspace}

\onehalfspacing


% Better looking empty set
\let\emptyset\varnothing

\newtheoremstyle{problem}{\topsep}{\topsep}%
{}%         Body font
{}%         Indent amount (empty = no indent, \parindent = para indent)
{\bfseries}% Thm head font
{\vspace{5pt}}%        Punctuation after thm head
{\newline}%     Space after thm head (\newline = linebreak)
{\thmname{#1}\thmnote{ #3}}%         Thm head spec

\theoremstyle{problem}
\newtheorem{prob}{Problem}

\theoremstyle{plain}
%\newtheorem{thm}{Theorem}
\newtheorem{lem}{Lemma}
\newtheorem{prop}{Proposition}
% No indent for whole page
\setlength\parindent{0pt}

\theoremstyle{remark}
\newtheorem{countex}{Counterexample}

\setlength{\columnsep}{1cm}
%%% Heading -- No need to edit %%%
\pagestyle{fancy}
\newcommand{\ceil}[1] {
  \left\lceil #1 \right\rceil
}

\newcommand{\abs}[1] {
  \left| #1 \right|
}

\newcommand{\myln}[1] {
  \ln{\left( #1 \right)}
}
% \input{~/latex-templates/custom.sty}

\rhead{
  Stefan Eng\\
  Dr. Katherine Stevenson\\
  Math 320\\
  12/4/13
}
\lhead{
  Analysis Worksheet 3
}

\cfoot{Page\ \thepage\ of\ \pageref{LastPage}}

%%%

\renewcommand{\qedsymbol}{$\blacksquare$}
% For manual qed box
\def\bs{\hspace{\stretch1}\ensuremath\blacksquare}

\begin{document}

%%% Make the title %%%
\begin{center}
  \textsc{\Large Foundations of Higher Mathematics}\\[.3cm]
  \textsc{\Large Homework 14 (Analysis 3)}
\end{center}
%%% End title %%%

%%% Start Assignment Here %%%
\begin{enumerate}
  % 1
\item Let $\langle a_n \rangle$ be defined recursively as follows: $a_1 = 1$ and $a_{n+1} = \sqrt{3a_n}$. 
%Use induction to prove that $a_n = 3^{\frac{1}{2} + \frac{1}{4} + \ldots + \frac{1}{2^{n-1}}}$.
\begin{proof}
  Let $S = \{n \in \mathbb N|\ n > 1 \text{ and } a_n = 3^{\frac{1}{2} + \frac{1}{4} + \ldots + \frac{1}{2^{n-1}}}\}$. We can see that $a_2 = \sqrt{3a_1} = \sqrt{3} = 3^{\frac{1}{2}}$. So $2 \in S$. Now assume that $a_k = 3^{\frac{1}{2} + \frac{1}{4} + \ldots + \frac{1}{2^{k-1}}}$.
  \begin{align*}
    a_{k+1} &= \sqrt{3 a_k}\\
    &= \sqrt{3}\ \sqrt{a_k}\\
    &= 3^{\frac{1}{2}} \cdot \sqrt{3^{\frac{1}{2} + \frac{1}{4} + \ldots + \frac{1}{2^{k-1}}}}\\
    &= 3^{\frac{1}{2}} \cdot 3^{\frac{1}{4} + \frac{1}{8} + \ldots + \frac{1}{2^{k-1}} + \frac{1}{2^k}}\\
    &= 3^{\frac{1}{2} + \frac{1}{4} + \frac{1}{8} + \ldots + \frac{1}{2^{k-1}} + \frac{1}{2^{(k+1)-1}}}
  \end{align*}
So $k + 1 \in S$. Therefore, by EPMI, $\{n \in \mathbb N|\ n > 1 \} \subseteq S$.
\end{proof}
% 2
\item Let $\langle a_n \rangle$ be defined as above. Prove that $\langle a_n \rangle$ is increasing and thus bounded below.
\begin{proof}
  Let $S = \{n \in \mathbb N|\ n \geq 1 \text{ and } a_{n+1} \geq a_n\}$. Observe that $1 \geq 1$ and $a_{n+1} = \sqrt{3a_n} = \sqrt{3} \geq 1$. So $1 \in S$. Assume $k \in S$, $a_{k+1} \geq a_k$.
\begin{align*}
  a_{k+1} &= \sqrt{3 a_k}  
\end{align*}
\end{proof}
 \begin{proof}
  Let $n < n + 1$. Want to show $a_n < a_{n+1}$. Then:
\begin{align*}
  a_n &= 3^{\frac{1}{2} + \frac{1}{4} + \ldots + \frac{1}{2^{n-1}}}\\
  &< 3^{\frac{1}{2^n}} \cdot 3^{\frac{1}{2} + \frac{1}{4} + \ldots + \frac{1}{2^{n-1}}}\\
  &< 3^{\frac{1}{2} + \frac{1}{4} + \ldots + \frac{1}{2^{n-1}} + \frac{1}{2^n}}\\
  &< a_{n+1}
\end{align*}
Therefore, $\langle a_n \rangle$ is increasing, and thus,
% Does this need a proof?
bounded below by $1$.
\end{proof}
% 3
\item Let $\langle a_n \rangle$ be defined as above. Prove that $\langle a_n \rangle$ is bounded above by 3.
\begin{proof}
  Let $S = \{n \in \mathbb N|\ a_n \leq 3\}$. Since $a_1 = 1$, and $1 \leq 3$, then $1 \in S$. Now assume that $k \in S$. Thus, $a_k \leq 3$. We want to show that $a_{k+1} \leq 3$.
  \begin{align*}
    a_{k+1} &= \sqrt{3a_k}\\
    &= \sqrt{3}\ \sqrt{a_k}\\
    &\leq \sqrt{3}\ \sqrt{3}\\
    &\leq 3
  \end{align*}
Thus, $k+1 \in S$. Therefore, by PMI, $S = \mathbb N$, so $\langle a_n \rangle$ is bounded above by $3$.
\end{proof}
% 4
\item $\langle a_n \rangle$ converges.
\begin{proof}
  $\langle a_n \rangle$ is increasing and $\{a_n|\ n \in \mathbb N \}$ is bounded above by 3, so $\langle a_n \rangle$ converges.
%Because this set is a subset of the real numbers, the completeness axiom tells us it has a least upper bound.
%Call this least upper bound $l$.
%Call $U$ the set of all upper bounds.
%In problem 3 we concluded that $3 \in U$.
%So $U$ is non-empty, and by the well-ordering principle there is a least element $l$.
%So, $a_n \leq l$. Let $\epsilon > 0$.
%It follows that $l - \epsilon < a_n \leq l$ since $l$ is the least upper bound.
%For $n > N$, we have $a_n \geq a_N$ because $a_n$ is increasing.
%It follows that, $l - \epsilon < a_N \leq a_n \leq l < l + \epsilon$ and thus, $\abs{a_n - l} < \epsilon$.
%Therefore $\langle a_n \rangle$ converges.
\end{proof}
% 5
\item $\displaystyle \lim_{n \to \infty} \frac{3n^2+n+4}{3 + n + 5n^2} = \frac{3}{5}$
\begin{proof}
  Observe that $\displaystyle \frac{3n^2+n+4}{3 + n + 5n^2} < \frac{3n^2+\frac{3}{5}n+\frac{9}{5}}{3 + n + 5n^2}$\\[0.5cm]
%Want to show that $\lim_{n \to \infty} \frac{3n^2+n+4}{3 + n + 5n^2} = \frac{3}{5}$.
Let $\epsilon > 0$.
It follows that
\begin{align*}
  \abs{\frac{3n^2+\frac{3}{5}n+\frac{9}{5}}{3 + n + 5n^2}} &= \abs{\frac{5(3n^2+\frac{3}{5}n+\frac{9}{5})}{5(3 + n + 5n^2)} - \frac{3(3 + n + 5n^2)}{5(3 + n + 5n^2)}}\\
  &= \abs{\frac{15n^2 + 3n + 9 - 15n^2 - 3n - 9}{5(3 + n + 5n^2)}}\\
  &= 0 < \epsilon
\end{align*}
Thus, it converges to $\displaystyle \frac{3}{5}$.\\[.5cm]
Now observe that $\displaystyle \frac{3n^2+n+4}{3 + n + 5n^2} > \frac{3n^2+n+4}{\frac{20}{3} + \frac{5}{3}n + 5n^2}$\\[.5cm]
Let $\epsilon > 0$. Then
\begin{align*}
\abs{\frac{3n^2+n+4}{\frac{20}{3} + \frac{5}{3}n + 5n^2} - \frac{3}{5}} &= \abs{\frac{5(3n^2 + n + 4) - 3(\frac{20}{3} + \frac{5}{3}n + 5n^2)}{5(\frac{20}{3} + \frac{5}{3}n + 5n^2)}}\\
&= \abs{\frac{15n^2 + 5n + 20 - 15n^2 - 5n - 20}{5(\frac{20}{3} + \frac{5}{3}n + 5n^2)}}\\
&= 0 < \epsilon
\end{align*}
Thus, it converges to $\displaystyle \frac{3}{5}$.

Therefore, by the squeeze theorem, $\displaystyle \frac{3n^2+n+4}{3 + n + 5n^2}$ converges to $\displaystyle \frac{3}{5}$.

\end{proof}

\end{enumerate}

\end{document}
