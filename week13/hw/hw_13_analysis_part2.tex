\documentclass{article}

\usepackage[utf8]{inputenc}
\usepackage[top=1.5in,left=.9in,right=.9in,bottom=1in,headheight=1in]{geometry}
\usepackage{fancyhdr}
\usepackage{lastpage}
\usepackage{amsmath,amssymb,amsthm,mathrsfs}
\usepackage{mathrsfs}
\usepackage{multicol}
\usepackage{enumerate}

\usepackage{setspace}

\onehalfspacing


% Better looking empty set
\let\emptyset\varnothing

\newtheoremstyle{problem}{\topsep}{\topsep}%
{}%         Body font
{}%         Indent amount (empty = no indent, \parindent = para indent)
{\bfseries}% Thm head font
{\vspace{5pt}}%        Punctuation after thm head
{\newline}%     Space after thm head (\newline = linebreak)
{\thmname{#1}\thmnote{ #3}}%         Thm head spec

\theoremstyle{problem}
\newtheorem{prob}{Problem}

\theoremstyle{plain}
%\newtheorem{thm}{Theorem}
\newtheorem{lem}{Lemma}
\newtheorem{prop}{Proposition}
% No indent for whole page
\setlength\parindent{0pt}

\theoremstyle{remark}
\newtheorem{countex}{Counterexample}

\newcommand{\ceil}[1] {
  \left\lceil #1 \right\rceil
}

\newcommand{\abs}[1] {
  \left| #1 \right|
}

\newcommand{\myln}[1] {
  \ln{\left( #1 \right)}
}

\setlength{\columnsep}{1cm}
%%% Heading -- No need to edit %%%
\pagestyle{fancy}

\rhead{
  Stefan Eng\\
  Dr. Katherine Stevenson\\
  Math 320\\
  12/4/13
}
\lhead{
  Analysis Worksheet 2
}

\cfoot{Page\ \thepage\ of\ \pageref{LastPage}}

%%%

\renewcommand{\qedsymbol}{$\blacksquare$}
% For manual qed box
\def\bs{\hspace{\stretch1}\ensuremath\blacksquare}

\begin{document}

%%% Make the title %%%
\begin{center}
  \textsc{\Large Foundations of Higher Mathematics}\\[.3cm]
  \textsc{\Large Homework 13 (Analysis 2)}
\end{center}
%%% End title %%%

%%% Start Assignment Here %%%
\begin{enumerate}
  % 1
\item Show that the sequence $<\frac{1}{n^p}>$ converges if $p > 1$. Find its limit.\\  
  Claim: $\displaystyle \lim_{n \to \infty} \frac{1}{n^p} = 0$
  \begin{proof}
    Let $\epsilon > 0$,
    We want $\abs{\frac{1}{n^p} - 0} = \abs{\frac{1}{n^p}} = \frac{1}{n^p} < \epsilon$.
    \begin{align*}
      \frac{1}{n^p} &< \epsilon\\
      1 &< n^p \epsilon\\
      \frac{1}{\epsilon} &< n^p\\
      \myln{\frac{1}{\epsilon}} &< p \myln{n}\\
      \frac{\myln{\frac{1}{\epsilon}}}{p} &< \myln{n}
    \end{align*}
    ...
    % This needs to break somewhere if p <= 1?????

    % Want to show that 1/n^p < epsilon for all n > M
    %Since $\epsilon > 0$, by the archimedian property, there exists an $M$ such that,
    %$\frac{1}{M} < \epsilon$.
    % Goal: 1/n^p < epsilon
    % Prove 1/n^p < 1/n
    % then 1/n < 1/M
    % => 1/n^p < epsilon
    % Put into align?
    %Observe that
    %$\frac{1}{n^p} < \frac{1}{n}$
    %$1 < \frac{n^p}{n}$
    %$1 < n^{p-1}$.
    %And since $p > 1$, clearly this inequality holds..     
  \end{proof}

  % 2
\item Show that the sequence $<\frac{2n}{3n+7}>$ converges. Find its limit.\\
  % Claim
  % Re right this in the "backwards way" where you pull N out of a hat
  Claim: $\displaystyle \lim_{n \to \infty} \frac{2n}{3n+7} = \frac{2}{3}$\\[.5cm]
  \begin{proof}
  Let $\epsilon > 0$, and observe that:
  \begin{align*}
    \left|\frac{2n}{3n+7} - \frac{2}{3}\right| &= 
    \left|\frac{3(2n) - 2(3n + 7)}{3(3n + 7)}\right|\\
    &= \left|\frac{-14}{9n+21}\right|\\
    &= \frac{14}{9n+21}
  \end{align*}
  We want $\displaystyle \frac{14}{9n+21} < \epsilon$. Solving for $\epsilon$ gives us:
  \begin{align*}
    \displaystyle \frac{14}{9n+21} &< \epsilon\\
    14 &< \epsilon(9n + 21)\\
    \frac{14}{\epsilon} &< 9n + 21\\
    \frac{14}{\epsilon} - 21 &< 9n\\
    \frac{14}{9 \cdot \epsilon} - \frac{7}{3} &< n
  \end{align*}
  So, choose $N$ such that $N = \displaystyle \left\lceil \frac{14}{9 \cdot \epsilon} - \frac{7}{3} \right\rceil$. Then for all $n > N$, we have $\displaystyle \left|\frac{2n}{3n+7} - \frac{2}{3}\right| < \epsilon$.
\end{proof}

% 3
\item (F\&P \#10) Prove that the sequence $<\frac{5n}{3n+1}>$ converges. \\
  Claim: $\displaystyle \lim_{n \to \infty} \frac{5n}{3n+1}$
  \begin{proof}
    Let $\epsilon > 0$. Choose $N \in \mathbb{N}$ such that $N = \displaystyle \ceil{\frac{1}{9\epsilon} - \frac{1}{3}}$. If $n > N$, then    
    \begin{align*}
      n &> \ceil{\frac{1}{9\epsilon} - \frac{1}{3}}\\
      n &> \frac{1}{9\epsilon} - \frac{1}{3}\\
      9n &> \frac{1}{\epsilon} - 3\\
      9n + 3 &> \frac{1}{\epsilon}\\
      \frac{1}{9n+3} &< \epsilon\\
      \abs{\frac{1}{9n+3}} &< \epsilon\\
      \abs{\frac{15n - 15n + 1}{9n+3}} &< \epsilon\\
      \abs{\frac{5n}{3n+1} - \frac{5}{3}} &< \epsilon
    \end{align*}
\end{proof}
  % 4
\item (F\&P \#6) Let $<x_n>$ be a sequence with the property that there is a real number $A$ and a natural number $N$ such that $x_n = A$ for all $n > N$. Prove that $<x_n> \to A$.

  % 5
\item (F\&P \#7) Show that $[0,1]$ is a neighborhood of $\frac{7}{8}$.
  \begin{proof}
    Let $\epsilon = \frac{1}{8}$. 
    Take $x \in (\frac{7}{8} - \epsilon, \frac{7}{8} + \epsilon)$.
    It follows that $ \frac{6}{8} < x < 1$. 
    Thus, $0 \leq x \leq 1$ so $x \in [0,1]$. 
    Therefore, $[0,1]$ is a neighborhood of $\frac{7}{8}$.
  \end{proof}
  % 6
\item Using no negative words say what it would mean for a sequence $<a_n>$ not to converge. Then, show that $<3^{2n-1}>$ does not converge.
  % Hints
  % a) Take an arbitrary $M \in \mathbb{R}$
  % b) Explain why $3^{2N-1} \geq M$ if and only if 
  %    (2N - 1)log_{10}(3) > log_{10}(M).
  % c) Use this to get an express of the form $3^{2N-1} \geq M$ if and only if
  %    $N \geq$ (an expression involving $M$ and logs).
  % d) Now choose $N$ accordingly.

  
\end{enumerate}

\end{document}
