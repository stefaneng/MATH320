\documentclass{article}

\usepackage[utf8]{inputenc}
\usepackage[top=1.5in,left=.9in,right=.9in,bottom=1in,headheight=1in]{geometry}
\usepackage{fancyhdr}
\usepackage{lastpage}
\usepackage{amsmath,amssymb,amsthm,mathrsfs}
\usepackage{mathrsfs}
\usepackage{multicol}
\usepackage{enumerate}

\usepackage{setspace}

\onehalfspacing


% Better looking empty set
\let\emptyset\varnothing

\newtheoremstyle{problem}{\topsep}{\topsep}%
{}%         Body font
{}%         Indent amount (empty = no indent, \parindent = para indent)
{\bfseries}% Thm head font
{\vspace{5pt}}%        Punctuation after thm head
{\newline}%     Space after thm head (\newline = linebreak)
{\thmname{#1}\thmnote{ #3}}%         Thm head spec

\theoremstyle{problem}
\newtheorem{prob}{Problem}

\theoremstyle{plain}
%\newtheorem{thm}{Theorem}
\newtheorem{lem}{Lemma}
\newtheorem{prop}{Proposition}
% No indent for whole page
\setlength\parindent{0pt}

\theoremstyle{remark}
\newtheorem{countex}{Counterexample}

\newcommand{\ceil}[1] {
  \left\lceil #1 \right\rceil
}

\newcommand{\abs}[1] {
  \left| #1 \right|
}

\newcommand{\myln}[1] {
  \ln{\left( #1 \right)}
}

\setlength{\columnsep}{1cm}
%%% Heading -- No need to edit %%%
\pagestyle{fancy}

\rhead{
  Stefan Eng\\
  Dr. Katherine Stevenson\\
  Math 320\\
  12/4/13
}
\lhead{
  Analysis Worksheet 2
}

\cfoot{Page\ \thepage\ of\ \pageref{LastPage}}

%%%

\renewcommand{\qedsymbol}{$\blacksquare$}
% For manual qed box
\def\bs{\hspace{\stretch1}\ensuremath\blacksquare}

\begin{document}

%%% Make the title %%%
\begin{center}
  \textsc{\Large Foundations of Higher Mathematics}\\[.3cm]
  \textsc{\Large Homework 13 (Analysis 2)}
\end{center}
%%% End title %%%

%%% Start Assignment Here %%%
\begin{enumerate}
  % 1
\item Show that the sequence $\langle \frac{1}{n^p} \rangle$ converges if $p > 1$. Find its limit.\\  
  Claim: $\displaystyle \lim_{n \to \infty} \frac{1}{n^p} = 0$
  \begin{proof}
    Let $\epsilon > 0$,
    We want $\abs{\frac{1}{n^p} - 0} = \abs{\frac{1}{n^p}} = \frac{1}{n^p} < \epsilon$.
    \begin{align*}
      \frac{1}{n^p} &< \epsilon\\
      1 &< n^p \epsilon\\
      \frac{1}{\epsilon} &< n^p\\
      \myln{\frac{1}{\epsilon}} &< p \myln{n}\\
      \frac{\myln{\frac{1}{\epsilon}}}{p} &< \myln{n}
    \end{align*}
    ...
    % This needs to break somewhere if p <= 1?????

    % Want to show that 1/n^p < epsilon for all n > M
    %Since $\epsilon > 0$, by the archimedian property, there exists an $M$ such that,
    %$\frac{1}{M} < \epsilon$.
    % Goal: 1/n^p < epsilon
    % Prove 1/n^p < 1/n
    % then 1/n < 1/M
    % => 1/n^p < epsilon
    % Put into align?
    %Observe that
    %$\frac{1}{n^p} < \frac{1}{n}$
    %$1 < \frac{n^p}{n}$
    %$1 < n^{p-1}$.
    %And since $p > 1$, clearly this inequality holds..     
  \end{proof}

  % 2
\item Show that the sequence $\langle \frac{2n}{3n+7} \rangle$ converges. Find its limit.\\
  % Claim
  % Re right this in the "backwards way" where you pull N out of a hat
  Claim: $\displaystyle \lim_{n \to \infty} \frac{2n}{3n+7} = \frac{2}{3}$\\[.5cm]
  \begin{proof}
  Let $\epsilon > 0$. Choose $N \in \mathbb N$ such that $N = \displaystyle \ceil{\frac{14}{9\epsilon} - \frac{7}{3}}$. It follows that:
  \begin{align*}
    N &\geq \frac{14}{9\epsilon} - \frac{7}{3}\\
    N - \frac{7}{3} &\geq \frac{14}{9\epsilon}\\
    9\epsilon(N - \frac{7}{3}) &\geq 14\\
    \epsilon &\geq \frac{14}{9N + 21}
  \end{align*}
  Therefore, if $n > N$, then
  \begin{align*}
    \abs{\frac{2n}{3n+7} - \frac{2}{3}} &= \abs{\frac{3(2n)}{3(3n+7)} - \frac{2(3n+7)}{3(3n+7)}}\\
    &= \abs{\frac{-14}{9n+21}}\\
    &= \frac{14}{9n+21} < \frac{14}{9N + 21} \leq \epsilon
  \end{align*}
  and so $\langle \frac{2n}{3n+7} \rangle$ converges to $\frac{2}{3}$.
  % \begin{align*}
  %   \left|\frac{2n}{3n+7} - \frac{2}{3}\right| &= 
  %   \left|\frac{3(2n) - 2(3n + 7)}{3(3n + 7)}\right|\\
  %   &= \left|\frac{-14}{9n+21}\right|\\
  %   &= \frac{14}{9n+21}
  % \end{align*}
  % We want $\displaystyle \frac{14}{9n+21} < \epsilon$. Solving for $\epsilon$ gives us:
  % \begin{align*}
  %   \displaystyle \frac{14}{9n+21} &< \epsilon\\
  %   14 &< \epsilon(9n + 21)\\
  %   \frac{14}{\epsilon} &< 9n + 21\\
  %   \frac{14}{\epsilon} - 21 &< 9n\\
  %   \frac{14}{9 \cdot \epsilon} - \frac{7}{3} &< n
  % \end{align*}
  % So, choose $N$ such that $N = \displaystyle \left\lceil \frac{14}{9 \cdot \epsilon} - \frac{7}{3} \right\rceil$. Then for all $n > N$, we have $\displaystyle \left|\frac{2n}{3n+7} - \frac{2}{3}\right| < \epsilon$.
\end{proof}

% 3
\item (F\&P \#10) Prove that the sequence $\langle \frac{5n}{3n+1} \rangle$ converges. \\
  Claim: $\displaystyle \lim_{n \to \infty} \frac{5n}{3n+1} = \frac{5}{3}$
  \begin{proof}
    Let $\epsilon > 0$. Choose $N \in \mathbb N$ such that $N = \displaystyle \ceil{\frac{5}{9\epsilon} - \frac{1}{3}}$. Thus,
    \begin{align*}
      N &\geq \frac{5}{9\epsilon} - \frac{1}{3}\\
      N + \frac{1}{3} &\geq \frac{5}{9\epsilon}\\
      \frac{9N + 3}{5} &\geq \frac{1}{\epsilon}\\
      \epsilon &\geq \frac{5}{9N + 3}
    \end{align*}
    Therefore, if $n > N$, then
    \begin{align*}
      \abs{\frac{5n}{3n+1} - \frac{5}{3}} &= \abs{\frac{3(5n)}{3(3n+1)} - \frac{5(3n+1)}{3(3n+1)}}\\
      &= \abs{\frac{15n - 15n - 5}{9n+3}}\\
      &= \abs{\frac{-5}{9n+3}}\\
      &= \frac{5}{9n+3} < \frac{5}{9N+3} \leq \epsilon
    \end{align*}
    and thus $\langle \frac{5n}{3n+1} \rangle$ converges to $\displaystyle \frac{5}{3}$.
    % Let $\epsilon > 0$. Choose $N \in \mathbb{N}$ such that $N = \displaystyle \ceil{\frac{1}{9\epsilon} - \frac{1}{3}}$. If $n > N$, then    
    % \begin{align*}
    %   n &> \ceil{\frac{1}{9\epsilon} - \frac{1}{3}}\\
    %   n &> \frac{1}{9\epsilon} - \frac{1}{3}\\
    %   9n &> \frac{1}{\epsilon} - 3\\
    %   9n + 3 &> \frac{1}{\epsilon}\\
    %   \frac{1}{9n+3} &< \epsilon\\
    %   \abs{\frac{1}{9n+3}} &< \epsilon\\
    %   \abs{\frac{15n - 15n + 1}{9n+3}} &< \epsilon\\
    %   \abs{\frac{5n}{3n+1} - \frac{5}{3}} &< \epsilon
    % \end{align*}
\end{proof}
  % 4
\item (F\&P \#6) Let $\langle x_n \rangle$ be a sequence with the property that there is a real number $A$ and a natural number $N$ such that $x_n = A$ for all $n > N$. Prove that $\langle x_n \rangle \to A$.
\begin{proof}
  Let $\epsilon = 0$. When $n > N$, then 
  \begin{align*}
    \abs{x_n - A} &= \abs{A - A}\\
    &= 0 < \epsilon
  \end{align*}
  Therefore, $x_n$ converges to $A$.
\end{proof}

  % 5
\item (F\&P \#7) Show that $[0,1]$ is a neighborhood of $\frac{7}{8}$.
  \begin{proof}
    Let $\epsilon = \frac{1}{8}$. 
    Take $x \in (\frac{7}{8} - \epsilon, \frac{7}{8} + \epsilon)$.
    It follows that $ \frac{6}{8} < x < 1$. 
    Thus, $0 \leq x \leq 1$ so $x \in [0,1]$.
    So, $(\frac{7}{8} - \epsilon, \frac{7}{8} + \epsilon) \subseteq [0,1]$.
    Therefore, $[0,1]$ is a neighborhood of $\frac{7}{8}$.
  \end{proof}
  % 6
\item Using no negative words say what it would mean for a sequence $\langle a_n \rangle$ not to converge. Then, show that $\langle 3^{2n-1} \rangle$ does not converge.\\
  % Is this right?
  %% 
% Stack overflow answer
%   For each real number $A$, there exists and $\epsilon > 0$ such that for all $N \in \mathbb N$, there exists an $n > N$ such that $|a_n - A| \geq \epsilon$.
%   What you want is a negation of "$\langle a_n \rangle$ converges".

% This means that there is no $A \in \mathbb{R}$ such that [some other conditions]. This means that for all $A \in \mathbb{R}$ those conditions are false.

% Those conditions are essentially "for all distances $\epsilon > 0$, the tail end of the sequence is $\epsilon$ away from $A$. The negation of this is that there is a distance $\epsilon$ such that no tail is $\epsilon$ away from it. Symbolically, $\exists \epsilon > 0$ such that $\forall N \in \mathbb{N}$, there is some $n > N$ such that $|A - a_n| \ge \epsilon$.

% ---

% A symbolic way to look at this is knowing how to negate $\forall$ and $\exists$. Let $P$ be a proposition. We claim that
% $$\lnot(\forall x, P) \iff \exists x, \lnot P$$
% The left hand side says "the claim that $P$ is true for all $x$ is false". The right hand side says "there is some $x$ for which $P$ is false". Similarly,
% $$\lnot(\exists x, P) \iff \forall x, \lnot P$$

% Applying this to the definition of convergence:
% $$\lnot (\langle a_n \rangle \textrm{ converges}) \iff \lnot( \exists A \in \mathbb{R} \ (\forall \epsilon > 0 \ ( \exists N \in \mathbb{N} \ ( \forall n > N, \ |A - a_n| < \epsilon))))$$
% $$\iff \forall A \in \mathbb{R} \ (\lnot(\forall \epsilon > 0 \ ( \exists N \in \mathbb{N} \ ( \forall n > N, \ |A - a_n| < \epsilon))))$$
% $$\iff \forall A \in \mathbb{R} \ (\exists \epsilon > 0 \ (\lnot( \exists N \in \mathbb{N} \ ( \forall n > N, \ |A - a_n| < \epsilon))))$$
% $$\iff \forall A \in \mathbb{R} \ (\exists \epsilon > 0 \ ( \forall N \in \mathbb{N} \ (\lnot( \forall n > N, \ |A - a_n| < \epsilon))))$$
% $$\iff \forall A \in \mathbb{R} \ (\exists \epsilon > 0 \ ( \forall N \in \mathbb{N} \ ( \exists n > N, \ \lnot( |A - a_n| < \epsilon))))$$
% $$\iff \forall A \in \mathbb{R} \ (\exists \epsilon > 0 \ ( \forall N \in \mathbb{N} \ ( \exists n > N, \ |A - a_n| \ge \epsilon)))$$

% In words: for all possible points $A$, there is some distance $\epsilon$ such that no matter how far out $(N)$ one cuts the sequence, there is some element $(n)$ that is too far away from $A$ (i.e., $|A - a_n| \ge \epsilon$).
%%
  % Hints
  % a) Take an arbitrary $M \in \mathbb{R}$
  % b) Explain why $3^{2N-1} \geq M$ if and only if 
  %    (2N - 1)log_{10}(3) > log_{10}(M).
  % c) Use this to get an express of the form $3^{2N-1} \geq M$ if and only if
  %    $N \geq$ (an expression involving $M$ and logs).
  % d) Now choose $N$ accordingly.

  
\end{enumerate}

\end{document}
