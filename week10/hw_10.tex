\documentclass{article}

\usepackage[utf8]{inputenc}
\usepackage[top=1.5in,left=.9in,right=.9in,bottom=1in,headheight=1in]{geometry}
\usepackage{fancyhdr}
\usepackage{lastpage}
\usepackage{amsmath,amssymb,amsthm}
\usepackage{mathrsfs}
\usepackage{multicol}
\usepackage{enumerate}



% Better looking empty set
\let\emptyset\varnothing

\newtheoremstyle{problem}{\topsep}{\topsep}%
{}%         Body font
{}%         Indent amount (empty = no indent, \parindent = para indent)
{\bfseries}% Thm head font
{\vspace{5pt}}%        Punctuation after thm head
{\newline}%     Space after thm head (\newline = linebreak)
{\thmname{#1}\thmnote{ #3}}%         Thm head spec

\theoremstyle{problem}
\newtheorem{prob}{Problem}

\theoremstyle{plain}
%\newtheorem{thm}{Theorem}
\newtheorem{lem}{Lemma}
\newtheorem{prop}{Proposition}
% No indent for whole page
\setlength\parindent{0pt}

\theoremstyle{remark}
\newtheorem{countex}{Counterexample}

\setlength{\columnsep}{1cm}
%%% Heading -- No need to edit %%%
\pagestyle{fancy}

\rhead{
  Stefan Eng\\
  Dr. Katherine Stevenson\\
  Math 320\\
  11/13/13
}
\lhead{
  Chapter 5 Worksheet
}

\cfoot{Page\ \thepage\ of\ \pageref{LastPage}}

%%%

\renewcommand{\qedsymbol}{$\blacksquare$}
% For manual qed box
\def\bs{\hspace{\stretch1}\ensuremath\blacksquare}

\begin{document}

%%% Make the title %%%
\begin{center}
\textsc{\Large Foundations of Higher Mathematics}\\[.3cm]
\textsc{\Large Homework 10}
\end{center}
%%% End title %%%

%%% Start Assignment Here %%%
\begin{enumerate}
  % Problem 1
  \item 
    \begin{enumerate} 
    \item $f(x) = 3x - 5$\\
      \textbf{Injective}: Let $f(x) = f(y)$. Then $3x - 5 = 3y - 5$. It follows that:
      \begin{align*}
        3x &= 3y\\
        x &= y
      \end{align*}\\
      \textbf{Surjective}: Let $y = f(x)$. It follows that $y = 3x - 5$. Thus:
      \begin{align*}
        y &= 3x - 5\\
        y - 5 &= 3x\\
        \frac{y - 5}{3} &= x
      \end{align*}
      So, $f(\frac{y-5}{3}) = y$ is a solution.
    \item $f(x) = x^2 + 5$\\
      \textbf{Not injective}: Let $a = 1$ and $b = -1$. We can see that $a \not = b$, but $f(a) = 1^2 + 5 = 6 = f(b)$.\\
      \textbf{Not surjective}: Take $f(x) = 0$. 
      \begin{align*}
        x^2 + 5 &= 0\\
        x^2 &= - 5
      \end{align*}
      Since $x$ has no solutions in $\mathbb{R}$, $f$ is not a surjection.
    \item $f(x) = x^2 - 5x - 6$\\
      \textbf{Not injective}: Let $a = 6$ and $b = -1$. Clearly $a \not = b$, but:
      \begin{align*}
        f(x) &= x^2 - 5x - 6\\
        &= (x - 6)(x + 1)\\
      \end{align*}
      Thus, $f(6) = 0 = f(-1)$.\\

      \textbf{Not Surjective}: Notice that $f'(x) = 2x - 5$. There is a critical point at $x = \frac{5}{2}$. (Minimum) It follows that $f(\frac{5}{2}) = -\frac{49}{4}$. Any $y$ value less than this value is impossible, and hence does not have a corresponding $x$ value. Thus, $f$ is not surjective.

    \item $f(x) = x^3 - 5$\\
      \textbf{Injective}: Let $f(x) = f(y)$ it follows that:
      \begin{align*}
        x^3 - 5 &= y^3 - 5\\
        x^3 &= y^3\\
        x &= y
      \end{align*}\\
      \textbf{Surjective}: Let $y = f(x)$. It follows that $y = x^3 - 5$, so a solution is $x = \displaystyle \sqrt[3]{y + 5}$

      % E
    \item $f(x) = x^3 - x$\\
      \textbf{Not injective}. Let $x = 1$ and $y = -1$. We can see that $x \not = y$, but $f(x) = 1 - 1 = 0$ and $f(y) = -1 + 1 = 0$. Thus, $f$ is not an injection.\\
      \textbf{Surjective} I graphed the function, and the function is continuous, and extends to all value of $y$.
    \end{enumerate}

    % Problem 2
    \item (F\&P {\small \#}4)\\[-.7cm]
      \begin{proof}
        Assume $f$ is strictly increasing. 
        We want to show $f$ is one-to-one. 
        Take an arbitary $x$ and $y$, $x \not = y$. 
        Assume $x$ is the smaller value.
        Since $f$ is strictly increasing and $x < y$, it follows that $f(x) < f(y)$.
        If $y$ is the smaller value, 
        swapping $x$ and $y$ in the previous argument we see that $f(y) < f(x)$.
        Thus, $f(x) \not = f(y)$. 
        So $f$ is a one-to-one function. 

        Now we want to show that $f^{-1}$ is strictly increasing. 
%        Since $f$ is strictly increasing,
%        Take $g = f^{-1}$.
        Let $a$ and $b$ be arbitrary elements of $\mathbb{R}$ such that $f(a) < f(b)$
        It follows that $a < b$ from the fact that $f$ is strictly increasing.
        Assume $f^{-1}(c) < f^{-1}(d)$.
        We can see that $f(f^{-1}(c)) = c$, and $f(f^{-1}(d)) = d$.
        It follows that $c < d$, since $f$ is strictly increasing.
        Therefore, $f^{-1}$ is strictly increasing.
      \end{proof}

      % Problem 3
    \item (F\&P {\small \#}7)
      \begin{enumerate} 
        % A
      \item One-to-one but not onto: $f(n) = 5n$
        % B
      \item Onto but not one-to-one: $f(n) = \begin{cases}
          n & \text{if n is even}\\
          n-1 & \text{if n is odd}
        \end{cases}$
        % C
      \item Neither onto nor one-to-one: $f(n) = n \mod 2$
        % D
      \item $X = \{1,2,3,4\}$\\
        There is \textit{not} a function that is one-to-one that does not map $X$ onto $X$ because one-to-one implies a unique pairing of elements in the domain to the codomain. Since $X$ is mapping to itself, every element will be mapped onto. There is \textit{not} a function that maps X onto X that is not one-to-one from similar logic. Every element in the range must be mapped onto the same set. 
      \end{enumerate}
     
      % Problem 4
    \item (F\&P {\small\#}73)
      \begin{enumerate} % Let $f: X \to Y$, let $A \subseteq X$, and let $B \subseteq Y$.
        \item Prove that if $y \in B$, then $f^{-1}(y) \subseteq f^{-1}(B)$.
          \begin{proof} Assume $y \in B$. It follows that $ y \in Y$ because $B \subseteq Y$. Also assume that $x \in f^{-1}(y)$. Want to show $x \in f^{-1}(B)$ (or $f(x) \in B$). It follows that $f(x) \in \{y\}$. Thus, $f(x) = y$. It follows that $f(x) = y \in B$, or $x \in f^{-1}(B)$.
          \end{proof}
        \item Prove that if $f(x) \in f(A)$ and $f$ is a one-to-one map, then $x \in A$.
          \begin{proof}
            Assume $f(x) \in f(A)$ and $f$ is injective. By Theorem 5.11 (part d), $A = f^{-1}(f(A))$, since $f$ is one-to-one. It follows that $x \subseteq f^{-1}(f(x))$. So, since $f(x) \in f(A)$, and $A = f^{-1}(f(A))$, $x \in A$.
          \end{proof}
        \end{enumerate} 
      % Problem 5
    \item $Tr : M_2(\mathbb{R}) \to \mathbb{R}$\\
      \textbf{Non-Injective}: Let $A = 
      \begin{bmatrix}
        1 & 0\\
        0 & 1
      \end{bmatrix}$
      and $B =
      \begin{bmatrix}
        1 & 1\\
        0 & 1
      \end{bmatrix}$. $A \not = B$ but $Tr(A) = 2 = Tr(B)$ so $Tr$ is not injective.\\

      \textbf{Surjective}: Let y be an arbitrary element in $\mathbb{R}$ such that $y = Tr(A)$. It follows that $y = a_{11} + a_{22}$. One solution is, $A = 
      \begin{bmatrix}
        a_{11} & 0\\
        0 & a_{22}
      \end{bmatrix}$. So $Tr$ is surjective. 

      % Problem 6
    \item \textbf{Non-Injective}: Let $A = 
      \begin{bmatrix}
        1 & 1\\
        1 & 1
      \end{bmatrix}$
      and $B =
      \begin{bmatrix}
        2 & 2\\
        2 & 2
      \end{bmatrix}$. $A \not = B$ but $Det(A) = 0 = Det(B)$ so $Det$ is not injective.\\
      \textbf{Surjective}: Assume $y \in \mathbb{R}$ such that, $y = Det(A)$. By definition, a solution to $y = Det(A)$ is 
      $$A = 
      \begin{bmatrix}
        a_{11} & a_{12}\\
        a_{21} & a_{22}
      \end{bmatrix}$$
Where $y = a_{11}a_{22} - a_{21}a_{12}$.
      % Problem 7
    \item \textbf{Not injective}: Let $A = 
      \begin{bmatrix}
        -1 & 1\\
        -1 & 1
      \end{bmatrix}$ and $B = 
      \begin{bmatrix}
        -2 & 2\\
        -2 & 2
      \end{bmatrix}$ $A \not = B$, but $Char(A) = x^2 - 0x + 0 = Char(B)$, so $Char$ is not injective.\\      

           \textbf{Surjective?:} Take $p \in P_2$ such that $p = ax^2 + bx + c$. We want to see if every solution in $P_2$ has a corresponding element in $M_2(\mathbb{R})$.
           \begin{align*}
             ax^2 + bx + c &= x^2 - Tr(A)x + Det(A)\\
             (a-1)x^2 + (b+Tr(A))x + c - Det(A) &= 0             
           \end{align*}
           Then, to solve, we see if there are any solutions from part of the quadratic formula:
           \begin{align*}
             &  \displaystyle \sqrt{b^2 - 4ac}\\
             &= \displaystyle \sqrt{(b+Tr(A))^2 -4(a-1)(c-Det(A))}
           \end{align*}
I assume that for some $A$ this works... I may have just overlooked a simple counter example.
           % Problem 8
    \item \begin{proof}
        Assume $g \circ f$ is injective. 
        We want to show that $f$ is injective. 
        Assume an arbitrary $a,b \in X$ such that $f(a) = f(b)$. 
        Let's define $y = f(a) = f(b)$. 
        Now we want to show that $a = b$. 
        We know that $g \circ f$ is injective, so $g(f(a)) = g(f(b))$. 
        It follows that $g(y) = g(y)$. 
        Since $g \circ f$ in an injection, we can conclude $a = b$. 
        Therefore, $f$ is injective.
      \end{proof}
    \end{enumerate}




\end{document}