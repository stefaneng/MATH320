\documentclass{article}

\usepackage[utf8]{inputenc}
\usepackage[top=1.5in,left=.9in,right=.9in,bottom=1in,headheight=1in]{geometry}
\usepackage{fancyhdr}
\usepackage{lastpage}
\usepackage{amsmath,amssymb,amsthm}
\usepackage{mathrsfs}
\usepackage{multicol}
\usepackage{enumerate}



% Better looking empty set
\let\emptyset\varnothing

\newtheoremstyle{problem}{\topsep}{\topsep}%
{}%         Body font
{}%         Indent amount (empty = no indent, \parindent = para indent)
{\bfseries}% Thm head font
{\vspace{5pt}}%        Punctuation after thm head
{\newline}%     Space after thm head (\newline = linebreak)
{\thmname{#1}\thmnote{ #3}}%         Thm head spec

\theoremstyle{problem}
\newtheorem{prob}{Problem}

\theoremstyle{plain}
%\newtheorem{thm}{Theorem}
\newtheorem{lem}{Lemma}
\newtheorem{prop}{Proposition}
% No indent for whole page
\setlength\parindent{0pt}

\theoremstyle{remark}
\newtheorem{countex}{Counterexample}

\setlength{\columnsep}{1cm}
%%% Heading -- No need to edit %%%
\pagestyle{fancy}

\rhead{
  Stefan Eng\\
  Dr. Katherine Stevenson\\
  Math 320\\
  11/13/13
}
\lhead{
  Chapter 5 Worksheet
}

\cfoot{Page\ \thepage\ of\ \pageref{LastPage}}

%%%

\renewcommand{\qedsymbol}{$\blacksquare$}
% For manual qed box
\def\bs{\hspace{\stretch1}\ensuremath\blacksquare}

\begin{document}

%%% Make the title %%%
\begin{center}
\textsc{\Large Foundations of Higher Mathematics}\\[.3cm]
\textsc{\Large Homework 10}
\end{center}
%%% End title %%%

%%% Start Assignment Here %%%
\begin{enumerate}
  % Problem 1
  \item 
    \begin{enumerate} 
    \item $f(x) = 3x - 5$
    \item $f(x) = x^2 + 5$
    \item $f(x) = x^2 - 5x - 6$
    \item $f(x) = x^3 - 5$
    \item $f(x) = x^3 - x$        
    \end{enumerate}

    % Problem 2
    \item 
      \begin{proof}
        Assume $f$ is strictly increasing. 
        We want to show $f$ is one-to-one. 
        Take an arbitary $x$ and $y$, $x \not = y$. 
        Assume $x$ is the smaller value.
        Since $f$ is strictly increasing and $x < y$, it follows that $f(x) < f(y)$.
        If $y$ is the smaller value, 
        swapping $x$ and $y$ in the previous argument we see that $f(y) < f(x)$.
        Thus, $f(x) \not = f(y)$. 
        So $f$ is a one-to-one function. 

        Now we want to show that $f^{-1}$ is strictly increasing...
      \end{proof}

      % Problem 3
    \item (F\&P {\small\#}7)
      \begin{enumerate} 
        % A
      \item
        % B
      \item
        % C
      \item
      \end{enumerate}

      % Problem 4
    \item (F\&P {small\#}7)
      \begin{enumerate} % Let $f: X \to Y$, let $A \subseteq X$, and let $B \subseteq Y$.
        \item a
        \item b
      \end{enumerate} 
      % Problem 5
    \item $Tr : M_2(\mathbb{R}) \to \mathbb{R}$\\
      \textbf{Non-Injective}: Let $A = 
      \begin{bmatrix}
        1 & 0\\
        0 & 1
      \end{bmatrix}$
      and $B =
      \begin{bmatrix}
        1 & 1\\
        0 & 1
      \end{bmatrix}$. $A \not = B$ but $Tr(A) = 2 = Tr(B)$ so $Tr$ is not injective.\\

      % I DON'T THINK THIS IS RIGHT?
      \textbf{Surjective}: Let y be an arbitrary element in $\mathbb{R}$. Then $Tr(A) = a_{11} + a_{22}$. One solution is, $A = 
      \begin{bmatrix}
        a_{11} & 0\\
        0 & a_{22}
      \end{bmatrix}$. So $Tr$ is surjective. 

      % Problem 6
    \item \textbf{Non-Injective}: Let $A = 
      \begin{bmatrix}
        1 & 1\\
        1 & 1
      \end{bmatrix}$
      and $B =
      \begin{bmatrix}
        2 & 2\\
        2 & 2
      \end{bmatrix}$. $A \not = B$ but $Det(A) = 0 = Det(B)$ so $Det$ is not injective.\\
      \textbf{Surjective}: Assume $A \in M_2(\mathbb{R})$...
      % Problem 7
    \item 7
      % Problem 8
    \item \begin{proof}
        Assume $g \circ f$ is injective. 
        We want to show that $f$ is injective. 
        Assume an arbitrary $a,b \in X$ such that $f(a) = f(b)$. 
        Let's define $y = f(a) = f(b)$. 
        Now we want to show that $a = b$. 
        We know that $g \circ f$ is injective, so $g(f(a)) = g(f(b))$. 
        It follows that $g(y) = g(y)$. 
        Since $g \circ f$ in an injection, we can conclude $a = b$. 
        Therefore, $f$ is injective.
      \end{proof}
    \end{enumerate}




\end{document}